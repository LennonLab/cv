\documentclass[11pt]{article}

% Match Word's default text area (6.5" x 9")
\usepackage[letterpaper, top=1in, bottom=1in, left=1in, right=1in]{geometry}

% --- Linking & URL handling ---
\usepackage{xurl} % allow line breaks in long URLs
\usepackage{hyperref}
\hypersetup{
  breaklinks=true,   % allow wrapping of long links
  colorlinks=false   % use light blue boxes instead of colored text
}

% --- Other packages ---
\usepackage{etaremune}
\usepackage{natbib}
\usepackage{enumitem}
\setlist[itemize]{noitemsep, topsep=0pt}
\usepackage{parskip}   % Adds spacing between paragraphs, no indent
\linespread{1.3}       % Slightly increase line spacing
%\usepackage[none]{hyphenat} % (commented out: can cause overfull boxes)
\usepackage{array}
\usepackage{tabularx}
\setlength{\parindent}{0pt} % removes default indent
\usepackage{longtable}
\usepackage{ltablex}
\keepXColumns
\renewcommand{\refname}{Publications} % for citations
\renewcommand{\arraystretch}{1.3} % more space between rows in tables
\usepackage[utf8]{inputenc}  % For pdflatex
\usepackage[T5]{fontenc}     % For Vietnamese support

\begin{document}

% Centered header
\begin{center}
  {\LARGE \textbf{Jay-Terrence Lennon}}\\[0.5em]
  Department of Biology, Indiana University, Bloomington, Indiana 47405, USA \\
  %Phone: (812) 856-0962 \\
  Email: \href{mailto:lennonj@iu.edu}{lennonj@iu.edu} \\
  Lab website: \url{https://lennonlab.github.io} \\
  Google Scholar: \url{https://goo.gl/qx4hHR}
\end{center}

%\vspace{1em}
\section*{Annual update}
\vspace{-1em} 
This document is a streamlined version of my \href{https://lennonlab.github.io/assets/docs/Lennon_CV.pdf}{full CV}. I have also elaborated on a few activities and positions for context. 

\section*{Professional experience}
\vspace{-1.6em} % Adjust this value as neede
\noindent
\begin{tabularx}{\textwidth}{@{}l@{\hspace{2em}}X@{}}
2021--2026  & Chair, Climate Change and Microbes portfolio,  American Academy of Microbiology (AAM)\\
2025--2028  & Chair, Applied and Environmental Microbiology Scientific Unit, American Society for Microbiology (ASM) \\
\end{tabularx}
\vspace{-1em} % adjust as needed (e.g. -0.75em, -1.25em)
I have been actively engaged in advancing science and policy at the intersection of microbiology, climate change, and sustainability. While serving on the Governing Board of the AAM, I was appointed Chair of the \href{https://asm.org/academy/climate-change-and-microbes-scientific-portfolio}{\url{Scientific Portfolio}} on climate change and microbes. With a \$2 million budget, I asembled a task force that has fostered partnerships across academia, non-profits, foundations, and the private sector to strengthen the connection between science and policy. In this role, I co-chaired colloquia with international leaders on Earth system modeling and the health impacts of climate change. In terms of policy, I moderated a congressional briefing and delivered official statements to the Environmental Protection Agency. This fall, I will participate in the Advocacy Academy, a hands-on program that trains scientists to communicate effectively with policymakers, the media, and the public, culminating in a day of in-person meetings with congressional offices on Capitol Hill in Washington, DC. 

The Portfolio’s work has positioned me as a founding member of the newly established Microbial Sustainability Coalition, a global initiative that is uniting microbiological societies to identify and implement climate solutions. These efforts are now being carried forward through my role as the inaugural Chair of the Scientific Unit on Applied and Environmental Microbiology at the American Society for Microbiology (ASM). This new Unit seeks to mobilize microbiology to address global sustainability challenges. In this capacity, I am helping to define the scientific scope, which integrates sustainability priorities with core foundations in fields including ecology and evolutionary biology. In addition to guiding scientific programming for the ASM annual meeting, I am also engaged in building philanthropic partnerships and strengthening communication with foundations and industry groups.

\vspace{-0.5em} % reduce space before the section
\section*{Honors and awards}
\vspace{-1em} % Adjust this value as needed 
\noindent
\begin{tabularx}{\textwidth}{@{}l@{\hspace{2em}}X@{}}
2024        & Highly Cited Author, American Society for Microbiology (ASM) \\
2024        & Soil Stars, Applied Microbiology International (AMI) \\ 
\end{tabularx}
\vspace{-2em} % adjust as needed (e.g. -0.75em, -1.25em)

\section*{Book project}
\vspace{-0.25em} 
This year, I have been working with an editor at Princeton University Press. Recently, I submitted a proposal to write a book titled \textit{In Suspended Animation: The Science of Dormancy and a New Understanding of Time, Survival, and the Boundaries of life.}
\vspace{-0.25em} % Adjust this value as needed 

\textbf{Description:} \textit{Suspended animation} explores how organisms contend with the constraints of time. Time limits what can be achieved. It forces choice, leaves outcomes vulnerable to chance, and does not allow for do-overs. In response, life has evolved a remarkable countermeasure. By entering a reversible state of reduced metabolic activity, an organism can pause, conserve energy, delay aging, and endure even the most inhospitable conditions. This phenomenon, known as dormancy, adds degrees of freedom to the model of life and opens possibilities that would otherwise remain closed. Dormancy buffers individuals and entire populations against risk and uncertainty. Far from being a biological curiosity, dormancy is an ancient and widespread strategy that reshapes our understanding of survival, evolution, and the very definition of life. Drawing on insights from ecology, medicine, neuroscience, and complexity theory, \textit{Suspended Animation} offers a new perspective on how life weathers uncertainty, stretches the boundaries of existence, and challenges our deepest assumptions about time.

\section*{Publications}
\vspace{-0.25em} % Adjust this value as needed 
\begin{etaremune}

\item[] \textnormal{\underline{Preprints, In Review, and In Press:}}

\item Karakoç C Shoemaker WR, Lennon JT (2025) Evolutionary bioenergetics of sporulation. \textit{bioRxiv} doi:10.1101/2025.08.26.672491. \href{https://www.biorxiv.org/content/10.1101/2025.08.26.672491v1}{(link)} Submitted to \textit{Proceedings of the National Academy of Sciences}

\item Mueller EA, van der Elst L, Gumennik A, Lennon JT (2025) The Enterostat: a 3D-printed bioreactor for simulating gut microbiome dynamics. \textit{bioRxiv} doi:10.1101/2025.08.21.671663. \href{https://www.biorxiv.org/content/10.1101/2025.08.21.671663v1}{(link)}

\item Lennon JT, Bittleston LS, Chen Q, Cooper VS, Fernández J, Gilbert JA, Häggblom MM, Harper LV, Jansson JK, Jiao N, Kuurstra EM, Peixoto RS, Rappuoli R, Schembri MA, Ventosa A, Vullo DL, Zhang C, Nguyen NK (2025) Microbes without borders: uniting societies for climate action. 

\textit{On 23 September 2025, paper will be published in:}
\begin{itemize}
  \item \textit{mBio} doi.org/10.1128/mbio.02136-25
    \item \textit{Sustainable Microbiology}
    \item \textit{ISMEJ}
    \item \textit{FEMS Microbiology Ecology}
    \item \textit{Sustainable Microbiology}  
    \item \textit{Ocean-Land-Atmosphere Research}
    \item \textit{Microbiology Australia}
\end{itemize}

\item Janet K. Jansson, Avi I. Flamholz, Raquel Peixoto, Salles JF, Lennon JT, Rosado A, Sanders IR, Jacobsen CS, Makhalanyane T, Schadt C, Gilbert JA (2025) Heterotrophic respiration by soil microbes in a changing climate. In review at \textit{Nature Reviews Earth \& Environment}

\item Wang J, Hu A, Cui Y, Bercovici S, Lu X, Lennon JT, Soininen J, Liu Y, Jiao N (2025) Towards the chemogeography of dissolved organic matter in the global ocean. In review at \textit{Environmental Science \& Technology}

\item Gilbert JA, Peixoto RS, Scholz AH, Dominguez-Bello MG, Korsten L, Berg G, Singh B, Boetius A, Wang F, Greening C, Jansson J, Lennon JT, Souza V, Thomas T, Cowan D, Crowhter T, Nguyen N, Harper L, Haraoui LP, Ishaq SL, Redford K (2025) The IUCN Microbial Conservation Specialist Group (MCSG): A global framework for safeguarding  microbial biodiversity. In press at \textit{Nature Microbiology} doi:10.1038/s41564-025-02113-5. \href{https://www.nature.com/articles/s41564-025-02113-5}{(link)} 

\item Overcast I, Calderon-Sanou I, Creer S, Dominguez-Garcia V, Hagen O, Hickerson MI, Jörger Hickfang T, Krehenwinkel H, Lennon JT, Méndez L, Méndez M, Onstein R, Pereira H, Qin C, Winter M, Yu DW, Zurell D, Gillespie RG (2025) The distribution of genetic diversity in ecological communities: A unifying measure for monitoring biodiversity change. \textit{EcoEvoRxiv}. doi:10.32942/X2Z64W. \href{https://ecoevorxiv.org/repository/view/9169/}{(link)} In review at \textit{Conservation Letters})

\item Wang J, Hu A, Cui Y, Bercovici SK, Lennon JT, Soininen J, Liu Y, Jiao N (2025) Geographical patterns and drivers of dissolved organic matter in the global ocean. \textit{Research Square}. doi:10.21203/rs.3.rs-6624570/v1. \href{https://assets-eu.researchsquare.com/files/rs-6624570/v1/a75822e6-0649-4c95-919e-3ba016f0bc96.pdf?c=1747022553}{(link)}

\item Bogar G, Lennon JT, Vander Stel H, Evans SE (2025) Simple, rapid, and sensitive assay for the quantification of total polysaccharides to estimate extracellular polymeric substances (EPS) in soil. \textit{bioRxiv} doi:10.1101/2025.05.22.654594. \href{https://www.biorxiv.org/content/10.1101/2025.05.22.654594v1}{(link)} In review at \textit{Journal of Microbiological Methods}

\item Măgălie A, Marantos A, Schwartz DA, Marchi J, Lennon JT, Weitz JS (2024) Phage infection fronts trigger early sporulation and collective defense in bacterial populations. \textit{bioRxiv}. doi:10.1101/2024.05.22.595388. \href{https://www.biorxiv.org/content/10.1101/2024.05.22.595388v1.full.pdf}{(link)} In revision at \textit{ISMEJ}

\item Hill CA, McMullen JC, Lennon JT (2024) Nitrogen enrichment alters selection on rhizobial genes. \textit{bioRxiv}. doi:10.1101/2024.11.25.625319. \href{https://www.biorxiv.org/content/10.1101/2024.11.25.625319v1.full.pdf}{(link)} In revision at \textit{mSystems}

\item Hu A, Cui Y, Bercovici A, Tanentzap AJ, Lennon JT, Lin X, Yang Y, Liu Y, Osterholz H, Dong H, Lu Y, Jiao N, Wang J (2024) Photochemical processes drive thermal responses of dissolved organic matter in the dark ocean. \textit{bioRxiv}. doi:10.1101/2024.09.06.611638. \href{https://www.biorxiv.org/content/10.1101/2024.09.06.611638v1.full.pdf}{(link)} In revision at \textit{Nature Communications}

\item McGill B, Jarzyna M, Diaz R, Barnes C, Diaz FH, Economo E, French C, Hagen O, James H, Kivlin S, Lahiri S, Lennon JT, Mascarenhas R, Ohyama L, Rabosky DL, Zhu K, Hickerson M, Gillespie R. A call to develop a coherent discipline of biodiversity science to address global change. In review at \textit{BioScience}

\vspace{1em}
\item[] \textnormal{\underline{Published:}}

\item Webster KD, Lennon JT (2025) Dormancy in the origin, evolution, and persistence of life on Earth. \textit{Proceedings of the Royal Society B: Biological Sciences} 292: 20242035. \href{https://lennonlab.github.io/assets/publications/Webster_Lennon_2025.pdf}{(pdf)}

\item Mueller EA, Lennon JT (2025) Residence time structures microbial communities through niche partitioning. \textit{Ecology Letters} 28: e70093. \href{https://lennonlab.github.io/assets/publications/Mueller_Lennon_2025.pdf}{(pdf)}

\item Nevermann HD, Gros C, Lennon JT (2025) A game of life with dormancy. \textit{Proceedings of the Royal Society B} 292: 20242543. \href{https://lennonlab.github.io/assets/publications/Nevermann_etal_2025.pdf}{(pdf)}

\item Lennon JT, Lehmkuhl BK, Chen L, Illingworth M, Kuo V, Muscarella ME (2025) \\Resuscitation-promoting factor (Rpf) terminates dormancy among diverse soil bacteria.\\
\textit{mSystems} 10: 10.1128/msystems.01517-24. \href{https://lennonlab.github.io/assets/publications/Lennon_etal_2025b.pdf}{(pdf)}

\item Rappuoli R, Nguyen N, Bloom DE, Brooks CG, Egamberdieva D, Lawley TD, Morhard R, Mukhopadhyay A, Lennon JT, Peixoto RS, Silver PA, Stein LY (2025) Microbes can capture carbon and degrade plastic — why aren’t we using them more? \textit{Nature} 639: 864–866. \href{https://lennonlab.github.io/assets/publications/Rappuoli_etal_2025a.pdf}{(pdf)}

\item Zang Z, Zhang C, Park KJ, Schwartz DA, Podicheti R, Lennon JT, Gerdt JP (2025) Streptomyces secretes a siderophore that sensitizes competitor bacteria to phage infection. \textit{Nature Microbiology}. \href{https://lennonlab.github.io/assets/publications/Zang_etal_2025.pdf}{(pdf)}

\item Beattie GA, Edlund A, Esiobu N, Gilbert J, Nicolaisen MH, Jansson JK, Jensen P, Keiluwei M, Lennon JT, Martiny JBH, Minnisi VR, Newmann D, Peixoto R, Schadt C, van der Meer JR (2025) Soil microbiome interventions for carbon sequestration and climate mitigation. \textit{mSystems}. \href{https://lennonlab.github.io/assets/publications/Beattie_etal_2025.pdf}{(pdf)}

\item Waldrop MP, Ernakovich JG, Vishnivetskaya TA, Schaefer SR, Mackleprang R, Bara J, O'Brien JM, Winkel M, Barbato RA, Heffernan L, Leewis MC, Hewitt RE, Hultman J, Sun Y, Biasi C, Bradley JA, Liebner S, Ricketts MP, Muscarella ME, Schütte U, Abuah F, Whalen E, Timling I, Voight C, Taş N, Lloyd KG, Silganen HMP, Rivkina EM, Voříšková J, Tao J, Liang R, Lennon JT, Onstott TC (2025) Microbial ecology of permafrost soils: populations, processes, and perspectives. \textit{Permafrost and Periglacial Processes}. \href{https://lennonlab.github.io/assets/publications/Waldrop_etal_2025.pdf}{(pdf)}

\item Rappuoli R, Nguyen N, Bloom DE, Brooks CG, Egamberdieva D, Lawley TD, Morhard R, Mukhopadhyay A, Lennon JT, Peixoto RS, Silver PA, Stein LY (2025) Microbial solutions for climate change — Toward an economically resilient future. \textit{American Society for Microbiology}. \href{https://lennonlab.github.io/assets/publications/Rappuoli_etal_2025b.pdf}{(pdf)}

\item Lennon JT and 32 others (2025) Colloquium report: Water, waterborne
pathogens and public health: environmental drivers. American Society for Microbiology, Washington, DC. \href{https://asm.org/reports/water-waterborne-pathogens-and-public-health-envi}{(link)}

\item Lennon JT, Rappuoli R, Bloom DE, Brooks CG, Egamberdieva D, Lawley TD, Morhard R, Mukhopadhyay A, Nguyen N, Peixoto RS, Silver PA, Stein LY (2025) Microbial solutions for climate change require global partnership. \textit{mBio} 16: 10.1128/mbio.00778-25. \href{https://lennonlab.github.io/assets/publications/Lennon_etal_2025a.pdf}{(pdf)}

\item Peixoto R, Voolstra CR, Stein LY, Hugenholtz P, Salles JF, Amin SA, Häggblom M, Gregory A, Makhalanyane TP, Wang F, Agbodjato NA, Wang Y, Jiao N, Lennon JT, Ventosa A, Bavoil PM, Miller V, Gilbert JA (2024) Microbiology at the brink: a unified call for action against climate catastrophe. 

\textit{Published in:}
\begin{itemize}
  \item Nature Microbiology 9: 3084–3085 \href{https://lennonlab.github.io/assets/publications/Peixoto_etal_2024a.pdf}{(pdf)}
  \item Nature Communications 15: 9637 \href{https://lennonlab.github.io/assets/publications/Peixoto_etal_2024b.pdf}{(pdf)}
  \item Nature Reviews Microbiology 23: 1–2 \href{https://lennonlab.github.io/assets/publications/Peixoto_etal_2024c.pdf}{(pdf)}
  \item Nature Reviews Earth and Environment 6: 4–5 \href{https://lennonlab.github.io/assets/publications/Peixoto_etal_2024d.pdf}{(pdf)}
  \item ISMEJ 18: wrae219 \href{https://lennonlab.github.io/assets/publications/Peixoto_etal_2024e.pdf}{(pdf)}
  \item mSystems 11: e0141624 \href{https://lennonlab.github.io/assets/publications/Peixoto_etal_2024f.pdf}{(pdf)}
  \item Communications Biology 7: 1466 \href{https://lennonlab.github.io/assets/publications/Peixoto_etal_2024g.pdf}{(pdf)}
  \item Communications Earth and Environment 5: 672 \href{https://lennonlab.github.io/assets/publications/Peixoto_etal_2024h.pdf}{(pdf)}
  \item FEMS Microbiology Ecology 100: fiae144 \href{https://lennonlab.github.io/assets/publications/Peixoto_etal_2024i.pdf}{(pdf)}
  \item NPJ Biodiversity 3: 34 \href{https://lennonlab.github.io/assets/publications/Peixoto_etal_2024j.pdf}{(pdf)}
  \item NPJ Biofilms and Microbiomes 10: 122 \href{https://lennonlab.github.io/assets/publications/Peixoto_etal_2024k.pdf}{(pdf)}
  \item NPJ Sustainable Agriculture 2: 23 \href{https://lennonlab.github.io/assets/publications/Peixoto_etal_2024l.pdf}{(pdf)}
  \item NPJ Climate Action 3: 1–3 \href{https://lennonlab.github.io/assets/publications/Peixoto_etal_2024m.pdf}{(pdf)}
  \item Sustainable Microbiology 1: qvae029 \href{https://lennonlab.github.io/assets/publications/Peixoto_etal_2024n.pdf}{(pdf)}
\end{itemize}

\item Beattie GA, Cotrufo FM, Crowther TW, Edlund A, Salles JF, Gilbert JK, Jansson JK, Jensen PR, Lennon JT, Makhalanyane T, Martiny JBH, Newman DK, Stevenson M (2024) Soil microbial strategies for climate mitigation: Report from a climate action workshop in Las Vegas, Nevada, February 2024. \textit{Sustainable Microbiology} 1: qvae033. \href{https://lennonlab.github.io/assets/publications/Beattie_etal_2024.pdf}{(pdf)}

\item Wu W, Hsieh C, Logares R, Lennon JT, Liu H (2024) Ecological processes shaping highly connected bacterial communities along strong environmental gradients. \textit{FEMS Microbiology Ecology} 100: fiae146. \href{https://lennonlab.github.io/assets/publications/Wu_etal_2024.pdf}{(pdf)}

\item Hu A, Jang KS, Tanentzap AJ, Zhao W, Lennon JT, Liu J, Li M, Stegen JC, Choi M, Lu Y, Feng X, Wang J (2024) Thermal responses of dissolved organic matter under global change. \textit{Nature Communications} 15: 576. \href{https://lennonlab.github.io/assets/publications/Hu_etal_2024.pdf}{(pdf)}

\item Lennon JT, Abramoff RZ, Allison SD, Burckhardt RM, DeAngelis KM, Dunne JP, Frey SD, Friedlingstein P, Hawkes CV, Hungate BA, Khurana S, Kivlin SN, Levine N, Manzoni S, Martiny AC, Martiny JBH, Nguyen N, Rawat M, Talmy D, Todd-Browne K, Vogt M, Wieder WR, Zakem E (2024) Priorities, opportunities, and challenges for integrating microorganisms into Earth system models for climate change prediction. \textit{mBio} 15: e00455-24. \href{https://lennonlab.github.io/assets/publications/Lennon_etal_2024.pdf}{(pdf)}

\end{etaremune}

\section*{Grants and funding}

\vspace{-1.25em}
\noindent
\begin{tabularx}{\textwidth}{@{}l@{\hspace{2em}}X@{}}
Grants submitted in past year

2025--2026	& Pending: National Science Foundation (NSF) “Conference: A unifying framework for dormancy across scales in natural, managed, and engineered ecosystems” Co-PI, \$99,000\\

2025--2030	& Pending: National Institutes of Health (NIH) “Cellular dormancy and virus entrapment” PI, \$2,161,566 (advanced to second level of review at Council meeting)\\ 

2025--2028 & Department of Natural Resources (DNR) "Development and testing of microbial mitigation of coalbed methane emissions in Indiana: A geo-microbial-engineering approach” Co-PI, \$287,295 \\


\end{tabularx}

\section*{Invited keynote, symposium, and conference presentations}
\vspace{-1.25em} % Adjust this value as needed 
\noindent
\begin{longtable}{@{}p{3em}@{\hspace{1.5em}}p{0.87\textwidth}@{}}

2026 & Invited workshop speaker, “Coarse-graining microbial ecology: from genes to physiological strategies to communities across environments”, Kavli Institute of Theoretical Physics (KITP), Santa Barbara, CA, USA \\

2026 & Invited workshop speaker, "EMBO: Ecological theories for microbial ecology: integrating molecular data" Ancona, Italy\\
2025 & Invites speaker, "ASM-SIAT Symposium on Microbes in Biotechnology", Shenzhen Institutes of Advanced Technology (SIAT), Chinese Academy of Sciences (CAS)\\

2025 & Invited workshop speaker,  "Microbial communities: energetics and dynamics across space and time" National Institute for Theory and Mathematics in Biology (NITMB), Chicago, Illinois, USA\\

2025 & Invited symposium speaker,  "Fungal island biogeography" Mycological Society of America, Portland, Oregon, USA\\

2025 & Invited symposium speaker, “Climate change and health: from micro to macro”, MAC-EPID, University of Michigan, Ann Arbor, Michigan, USA \\

2025 & Keynote speaker, Applied and Environmental Sciences (AES) Retreat, American Society of Microbiology \\

2025 & Invited speaker, “Synthetic biology enabling tools”, American Society for Biochemistry and Molecular Biology, Chicago, IL, USA \\

2025 & Invited speaker, “Evolutionary ecology of dormancy in a community context”, Ecological Society of America, Baltimore, MD, USA \\

2025 & Invited speaker, Council on Microbial Sciences (COMS) meeting, American Society of Microbiology \\

2024 & Invited speaker, Union Session: “One Health, Microbes, and Geosciences”, American Geophysical Society, Washington, DC, USA \\

2024 & Plenary speaker, Chinese Association of Microbial Ecology, Qingdao, China \\

2024 & Plenary speaker, International Symposium on Soil Microbiomes and Soil Health, Yangling, China \\

2024 & Plenary speaker, mLife Research Conference, Shenzhen, China \\

2024 & Invited speaker, “Harnessing transformational technologies symposium”, Los Alamos National Laboratory and National Academies of Sciences, Santa Fe, NM, USA \\

2024 & Invited speaker, mSystems Thinking Series, “Critical concepts in microbial dormancy”, Virtual \\

2024 & Invited speaker, ASM Public and Scientific Affairs Committee (PSAC) Meeting, Virtual \\
\end{longtable}


\section*{Invited seminars}
\vspace{-1.25em} % Adjust this value as needed 
\noindent
\begin{longtable}{@{}p{3em}@{\hspace{1.5em}}p{0.87\textwidth}@{}}
2025 & University of Illinois, Department of Microbiology \\

2025 & University of Alaska Fairbanks, Department of Biology and Wildlife \\

2025 & University of Maryland, Department of Biology \\

2024 & Ocean University of China, Institute of Evolution and Marine Biodiversity \\

2024 & Northwest A\&F University, Department of Environmental Science and Engineering \\

2024 & University of Southern California, Department of Biological Sciences \\

2024 & University of California San Diego, Department of Ecology, Behavior, and Evolution \\

2024 & Carnegie Institute of Science \\

2024 & Pennsylvania State University, One Health Microbiome Center Seminar Series \\

2024 & Lehigh University, Earth \& Environmental Sciences \\
\end{longtable}

\section*{Organizer for symposia, workshops, and conferences}
\vspace{-1.25em} % Adjust this value as needed 
\noindent
\begin{longtable}{@{}p{3em}@{\hspace{1.5em}}p{0.87\textwidth}@{}}

2025 & Co-organizer, “Cancer dormancy and therapy resistance: from models to the clinic” Rome, Italy \\

2025 & Co-chair, colloquium steering committee “Microbes, human health, and climate change” American Academy of Microbiology, Washington, DC, USA \\

2024 & Steering Committee, “Enhancing methane mitigation strategies via methanogenesis and methanotrophy” Microbe meeting, ASM, Atlanta, Georgia, USA \\
\end{longtable}

\section*{Invited participant: workshops, round tables, and synthesis groups}
\vspace{-1.25em} % Adjust this value as needed 
\noindent
\begin{longtable}{@{}p{2cm}@{\hspace{1em}}p{14cm}@{}}
2025 & Invited participant, Schmidt Sciences "Biology for Greenhouse Gas Mitigation Workshop", Washington, DC, USA\\

2025 & Invited participant, “Microbes and climate change global strategy meeting” American Academy of Microbiology, Washington, DC, USA \\

2025 & Invited participant, "Microbial solutions for climate change - towards an economically sustainable future", American Society for Microbiology, Los Angeles, CA, USA \\

2024 & Invited colloquium participant, “Impacts of the changing climate on water, water-borne pathogens, and human health colloquium” American Academy of Microbiology and the American Geophysical Union, Washington, DC, USA \\

2024 & Invited participant, “Dormancy in soil microbiomes” Northwest University, Xi’An, China \\

2024 & Invited moderator, American Society of Microbiology, Meet the Policymaker Series: National Climate Assessment, virtual \\

2024 & Invited workshop participant, “Developing a rapid, cost-effective, and information-rich metric of biodiversity resilience” German Centre for Integrative Biodiversity Research (iDiv), Leipzig, Germany \\

2024 & Invited participant, “Soil microbial strategies for climate mitigation” Oath Soil Life, Las Vegas, NV, USA \\
\end{longtable}

\section*{Contributed presentations}
\vspace{-0.5em}

{\setlength{\parskip}{0.3em}  % Tightens vertical spacing between items

\hangindent=2em Lennon JT (2025) United States Hearing, Reconsideration of 2009 Endangerment finding and greenhouse gas vehicle standards. Docket No. EPA-HQ-OAR-2025-0194. \par

\hangindent=2em Lennon JT (2025) Resuscitation-promoting factor (Rpf) terminates dormancy among diverse soil bacteria. American Society for Microbiology, Los Angeles, California, USA. \par

\hangindent=2em Nevermann Henrik D, Gros C, Lennon JT (2025) A game of life with dormancy. Deutsche Physikalische Gesellschaft, DPG (German Physical Society), Regensburg, Germany. \par

\hangindent=2em Nevermann Henrik D, Gros C, Lennon JT (2025) A game of life with dormancy. Dynamical Systems Applied on Biology and Natural Sciences (DSABNS), Naples, Italy. \par

\hangindent=2em Nevermann Henrik D, Gros C, Lennon JT (2025) A game of life with dormancy. Dynamic Days, Bremen, Germany. \par
} 

\section*{Service}
\vspace{-0.5em}


\begin{longtable}{@{}p{3em}@{\hspace{3.5em}}p{0.87\textwidth}@{}}

2025 --     & Steering committee, International Union for Conservation (IUCN), Microbial Conservation Specialist Group (MCSG)\\

2024--     & A Global Partnership to Address Climate \& Biodiversity Crisis, American Society for Microbiology (ASM) and the International Union of Microbiological Societies (IUMS) \\

\end{longtable}

\vspace{-0.5em}
\textnormal{\underline{Editor:}}\\[-2.5em]
\begin{longtable}{@{}p{4.5em}@{\hspace{3.5em}}p{0.87\textwidth}@{}}
2021--2027 & Editorial Board, \textit{ISME Journal} (\textit{International Society for Microbial Ecology}) \\
\end{longtable}


\vspace{-1em}
\textnormal{\underline{Promotion and tenure letter-writer}}\\[-2.5em]
\begin{longtable}{@{}p{4em}@{\hspace{2em}}p{0.85\textwidth}@{}}
2025 & University of California San Diego,  Ecology, Behavior, and Evolution \\
2025 & University of Minnesota, Department of Ecology, Evolution, and Behavior \\
2025 & University of Miami, Department of Biology \\
2025 & Michigan Technological University, Department of Biological Sciences\\
2025 & Case Western Reserve University, Department of Biology \\
2025 & University of Tennessee, Department of Ecology and Evolutionary Biology \\
2025 & Purdue University, Department of Food Science \\
2025 & Weizman Institute of Science \\
2024 & Stanford University, Department of Biology \\
2024 & University of Arizona, Department of Environmental Science \\
2024 & University of Haifa, Department of Evolutionary and Environmental Biology \\
2024 & Pennsylvania State University, Department of Plant Sciences \\
2024 & North Carolina State University, Department of Biological Sciences \\
2024 & The Ohio State University, Department of Microbiology \\
2024 & University of Wisconsin, Department of Soil Science \\
2024 & Marine Biological Laboratory, Ecosystems Center \\
\end{longtable}

\vspace{-1em}
\textnormal{\underline{Professional societies}} \\[-2.5em]
\begin{longtable}{@{}p{4.5em}@{\hspace{3.5em}}p{0.87\textwidth}@{}}
2025--2026 & Chair, Academy Leadership Nomination Subcommittee, American Academy of Microbiology \\
2024--2025 & Scientific Program Leader, American Society for Microbiology \\
2024--2025 & Finance Committee, Ecological Society of America \\
2023--2024 & Strategic Planning Working Group, Ecological Society of America \\
2023--2025 & Audit Committee, Ecological Society of America \\
2022--2024 & Publications Committee, Ecological Society of America \\
\end{longtable}


\vspace{-1em}
\textnormal{\underline{University and College}} \\[-2.5em]

\begin{longtable}{@{}p{4.5em}@{\hspace{2em}}p{0.85\textwidth}@{}}
2025-- & Faculty Honors and Awards Committee, Department of Biology \\
2024--2025 & College Research Faculty Promotion Subcommittee \\
2024       & Search committee, Faculty 100 Initiative, Synthetic Biology \\
\end{longtable}

\vspace{-1.5em}
\section*{Academic advisees}
\vspace{-0.5em}

\textnormal{\underline{Postdocs}}\\[-2.5em]
\begin{longtable}{@{}p{4em}@{\hspace{2em}}p{0.85\textwidth}@{}}
2024--     & Emma Bueren \\
\end{longtable}

\vspace{-1em}
\textnormal{\underline{Graduate Students}} \\[-2.5em]

\begin{longtable}{@{}p{4em}@{\hspace{2em}}p{0.85\textwidth}@{}}
2025--     & Jitul Bora (Ph.D., Biology, Indiana University) \\
2024--     & Anna Lennon (Ph.D., Biology, Indiana University) \\
2024--     & El Park (Ph.D., Biology, Indiana University) \\

\end{longtable}

\end{document}
