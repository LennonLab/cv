\documentclass[11pt]{article}  % Required as first line

% Page and layout settings
\usepackage[letterpaper, top=1in, bottom=1in, left=1in, right=1in]{geometry}
\linespread{1.3}  % Slightly increase line spacing
\setlength{\parindent}{0pt}  % Removes paragraph indentation
\renewcommand{\arraystretch}{1.3}  % More space between table rows

% Font and encoding
\usepackage[utf8]{inputenc}  % For pdflatex
\usepackage[T5]{fontenc}     % For Vietnamese support

% Hyperlinks and URLs
\usepackage{hyperref}
\usepackage{url}

% Text formatting and layout
\usepackage{parskip}         % Adds spacing between paragraphs, no indent
\usepackage[none]{hyphenat}  % Disable hyphenation

% Tables
\usepackage{array}
\usepackage{tabularx}
\usepackage{longtable}
\usepackage{ltablex}
\keepXColumns

% Lists
\usepackage{enumitem}
\setlist[itemize]{noitemsep, topsep=0pt}

% Citations (optional — used only if you call \cite or \bibliography)
\usepackage{natbib}
% \usepackage{etaremune}  % Removed for non-numbered references

% Rename bibliography section if needed
\renewcommand{\refname}{Publications}


\begin{document}

% Centered header
\begin{flushleft}
  \textbf{Nominee name:} Jay-Terrence Lennon\\
  \textbf{Institution:} Indiana University, Bloomington \\
  \textbf{Position title:} Professor of Biology\\
  \textbf{Email address:} \href{mailto:lennonj@iu.edu}{lennonj@iu.edu} \\
  %Lab website: \url{https://lennonlab.github.io} \\
  %Lab wiki: \url{https://lennon.bio.indiana.edu} \\
  %Google Scholar: \url{https://goo.gl/qx4hHR}
\end{flushleft}

%\vspace{1em}

\section*{A. Education}
\noindent
\begin{tabular}{@{}l@{\hspace{3em}}l@{\hspace{3em}}l@{\hspace{3em}}l@{}}
1995 & B.S.    & Environmental Forest Biology     & SUNY-ESF at Syracuse \\
1999 & M.A.    & Ecology and Evolutionary Biology  & University of Kansas \\
2004 & Ph.D. & Ecology and Evolutionary Biology  & Dartmouth College \\
\end{tabular}


\section*{B. Positions and Honors}
\vspace{-0.1em}
\noindent\underline{\textbf{Positions}}\\[-2em]
\noindent

\begin{tabularx}{\textwidth}{@{}l@{\hspace{2em}}X@{}}
2025--2028  & Chair, Applied and Environmental Microbiology Scientific Unit, American Society for Microbiology (ASM) \\
2023         & Visiting Professor, Goethe University, Frankfurt, Germany \\
2023         & Short-term Visiting Professor, ETH Zürich, Centre for Origin and Prevalence of Life \\
2020--2024   & Faculty, Complex Networks and Systems, Indiana University \\
2018--2022   & Faculty, Microbial Diversity Course, Marine Biological Laboratory, Woods Hole \\
2016--       & Professor, Indiana University, Department of Biology; Core Faculty in Evolution, Ecology, and Behavior; Affiliated Faculty in Microbiology \\
2016--2017   & Whitman Center Associate, Marine Biological Laboratory, Woods Hole \\
2016--2017   & Visiting Professor, Montana State University, Department of Microbiology and Immunology \\
2012--2016   & Associate Professor, Indiana University, Department of Biology; Core Faculty in Evolution, Ecology, and Behavior; Affiliated Faculty in Microbiology \\
2012--2015   & Adjunct Professor, W.K. Kellogg Biological Station, Michigan State University \\
2012         & Associate Professor, W.K. Kellogg Biological Station, Department of Microbiology and Molecular Genetics, Michigan State University \\
2011--2016   & Ad hoc Graduate Faculty, Michigan Technological University \\
2008--2012   & Adjunct Professor, Plant Biology Department, Michigan State University \\
2006--2012   & Assistant Professor, W.K. Kellogg Biological Station, Department of Microbiology and Molecular Genetics, Michigan State University \\
2004--2006   & Postdoctoral Research Associate, Brown University, Department of Ecology and Evolutionary Biology \\
\end{tabularx}

\vspace{-1.5em} % Reduce space
\noindent\underline{\textbf{Honors}}\\[-2.5em] % Reduce space
\noindent

\begin{tabularx}{\textwidth}{@{}l@{\hspace{2em}}X@{}}
2024        & Highly Cited Author, American Society for Microbiology (ASM) \\
2024        & Soil Stars, Applied Microbiology International (AMI) \\ 
2023        & Humboldt Prize, Alexander von Humboldt Foundation \\
2022--2025  & Governing Board, Ecological Society of America (ESA) \\
2022--2027  & Chair, Climate Change Task Force, American Academy of Microbiology (AAM) \\
2021        & Fellow, Ecological Society of America (ESA) \\
2020--2022  & Distinguished Lecturer, American Society for Microbiology (ASMDL) \\
2020--2026  & Governor, American Academy of Microbiology (AAM) \\
2019        & Fellow, American Academy of Microbiology (AAM) \\
2019--2024  & Highly Cited Researcher, Clarivate, Cross-Field \\
2018        & Fellow, American Association for the Advancement of Science (AAAS) \\
2012        & Kavli Fellow, National Academy of Sciences \\
2004        & USDA National Research Initiative (NRI) Postdoctoral Fellowship Award \\
2004        & Hannah T. Croasdale Graduate Scholar Award, Dartmouth College \\
2004        & Milton L. Shifman Endowed Scholarship, Marine Biological Laboratory \\
2004        & Albert Cass Fellowship, The Rockefeller University \\
2004        & Nathan Jenks Biology Award, Dartmouth College \\
2003        & Best student presentation, North American Lake Management Society National Meeting, Mashantucket, CT \\
2002        & NSF Doctoral Dissertation Improvement Grant (DDIG) \\
1999--2004  & Dartmouth Fellowship, Dartmouth College \\
1995        & Undergraduate honors: \textit{Magna Cum Laude}; President’s List; Alpha Sigma Xi, SUNY-ESF \\
1992        & Outstanding history student, SUNY Oswego \\
\end{tabularx}

\section*{C. Publications}
10 most relevant of 148. For full list, see: \href{https://scholar.google.com/citations?hl=en&user=d-hWatsAAAAJ}{Google Scholar}, \href{https://www.ncbi.nlm.nih.gov/myncbi/jay.lennon.1/bibliography/public/}{PubMed}, or \href{https://www.ncbi.nlm.nih.gov/myncbi/jay.lennon.1/bibliography/public/}{Lab Website}

\vspace{-0.25em}

\begin{itemize}[leftmargin=*, label={}, itemsep=1em]

  \item[] \textnormal{\underline{Preprints:}}

  \item Overcast I, Calderon-Sanou I, Creer S, Dominguez-Garcia V, Hagen O, Hickerson MI, Jörger-Hickfang T, Krehenwinkel H, Lennon JT, Méndez L, Méndez M, Onstein R, Pereira H, Qin C, Winter M, Yu DW, Zurell D, Gillespie RG (2025) The distribution of genetic diversity in ecological communities: A unifying measure for monitoring biodiversity change. \textit{EcoEvoRxiv}. doi:10.32942/X2Z64W. \href{https://ecoevorxiv.org/repository/view/9169/}{(link)}

  \item Wang J, Hu A, Cui Y, Bercovici SK, Lennon JT, Soininen J, Liu Y, Jiao N (2025) Geographical patterns and drivers of dissolved organic matter in the global ocean. \textit{Research Square}. doi:10.21203/rs.3.rs-6624570/v1. \href{https://assets-eu.researchsquare.com/files/rs-6624570/v1/a75822e6-0649-4c95-919e-3ba016f0bc96.pdf?c=1747022553}{(link)}

  \item[] \textnormal{\underline{White papers:}}

  \item Rappuoli R, Nguyen N, Bloom DE, Brooks CG, Egamberdieva D, Lawley TD, Morhard R, Mukhopadhyay A, Lennon JT, Peixoto RS, Silver PA, Stein LY (2025) Microbial solutions for climate change — Toward an economically resilient future. \textit{American Society for Microbiology}. \href{https://lennonlab.github.io/assets/publications/Rappuoli_etal_2025b.pdf}{(pdf)}

\end{itemize}

\section*{Grants and Funding}

\$31,622,017


\vspace{-1.25em}
\noindent
\begin{tabularx}{\textwidth}{@{}l@{\hspace{2em}}X@{}}
2023--2024 & Department of Defense (DoD) “Molecular-based methods for the mark-recapture of microorganisms” PI, \$225,000 \\
2023 & Humboldt Research Fellowship, Alexander von Humboldt Foundation, Germany, €60,000 (\$68,780) \\
2022--2025 & Army Research Office (ARO) “Complexity of the gut microbiome: a quantitative and experimental approach” PI, \$449,862 \\
2020--2025 & National Science Foundation (NSF) “BII-Implementation: Multiscale interactions of nested genomes in symbiosis: From genes to global change” Co-investigator with R Whitaker, \$12.5M \\
2020--2023 & National Science Foundation (NSF) “Collaborative Research: BEE: A dormancy refuge in host-parasite eco-evolutionary dynamics” PI, \$976,617 \\
2020--2025 & National Science Foundation (NSF) “CNH2-L: Resilience to drought or a drought of resilience? The potential for interactions and feedbacks between human adaptation and ecological adaptation” Co-PI with J Lau, \$1,599,684 \\
2020--2023 & National Aeronautics and Space Administration (NASA) “Energy limitation and the evolution of microbial dormancy” PI, \$733,792 \\
2020 & Army Research Office (ARO) "Mechanistic insight into bacterial metabolism from long-term evolution experiments" PI, \$550,000 \\
2018--2020 & Army Research Office (ARO) “Microbial evolution: linking genes, phenotype, and fitness in bacterial populations” PI, \$499,330 \\
2018--2019 & Indiana University Collaborative Research Grant (IUCRG). “Complexity of the gut microbiome: an experimental approach” PI, \$74,990 \\
2018--2019 & National Aeronautics and Space Administration (NASA) “Microbial dormancy and adaptation to energy-limitation” Co-PI with W Shoemaker, \$4,997 \\
2018--2021 & National Science Foundation (NSF) “Microbiome influences on the development of sociality in uni- and bi-parental rodents” Senior Personnel with G Demas, \$699,573 \\
2017--2018 & Army Research Office (ARO) "Connecting phenotype to genotype in evolved prokaryotic populations.” Co-PI with J.B. McKinlay. \$197,390 \\
2015--2018 & National Science Foundation (NSF) “Dissertation Research: Metabolic resource partitioning: scaling microbial physiology from individual activity to ecosystem function” PI, \$19,004 \\
2015--2022 & Department of Defense, Multidisciplinary University Research Initiatives (MURI) Program, “Mechanisms of prokaryotic evolution” PI, \$6,248,455 \\
2015--2020 & National Science Foundation (NSF) “Dimensions: Collaborative Research: Microbial seed banks: processes and patterns of dormancy-driven biodiversity” PI, \$1,997,144 \\
2015--2016 & Indiana Academy of Science “Metabolic fate of terrestrial carbon resources: anabolic vs. catabolic processes” Co-PI with M Muscarella, \$2,200 \\
2013--2015 & Polish Ministry of Science and Higher Education “Interactive effects of multiple regulating factors on cladoceran species richness and community structure” Co-PI with A Dzialowski, \$128,000 \\
2012--2014 & Center for Water Sciences (CWS) and Environmental Science and Policy Program (ESPP), Michigan State University, “Building partnerships in water research between Grand Valley State University and Michigan State University: the molecular genetic basis for invasiveness of milfoils and a time-series observatory for investigating metabolism in Muskegon Lake” Co-PI with P. Ostrom, \$99,996 \\
2012--2015 & National Science Foundation (NSF) “Collaborative Research: PEATcosm: Understanding the interactions of climate, plant functional groups and carbon cycling in peatland ecosystems” Co-PI with E Kane, \$677,185 \\
2011--2014 & United States Department of Agriculture (USDA), Agriculture and Food Research Initiative (AFRI) “Microbial seed banks: patterns and mechanisms of bacterial dormancy in soils” PI, \$499,956 \\
2011--2014 & National Science Foundation (NSF), “Do biological processes result in the atmospheric 17O mass independent anomaly in nitrous oxide? Resolution and establishment of 17O as a tracer of microbial production.” Co-PI with N Ostrom \$677,366 \\
2011--2012 & Huron Mountain Wildlife Foundation (HMWF), “Browning of freshwater ecosystems: will terrestrial carbon loading alter the diversity and function of aquatic microbial communities?” PI, \$5,600 \\
2011--2012 & BEACON, Michigan State University, “Contemporary evolution of cyanobacteria and viruses: implications for marine nutrient cycling” PI, \$76,964 \\
2010--2013 & National Science Foundation (NSF), “Greenhouse facility to support field ecology and evolution research and teaching at the Kellogg Biological Station”, Co-PI with K Gross, \$200,000 \\
2010--2013 & National Science Foundation (NSF), “Field facilities improvements for terrestrial and aquatic ecology at the Kellogg Biological Station.” Co-PI with K. Gross, \$176,000 \\
2009--2012 & Polish Ministry of Sciences, “Zooplankton invasions: how do local and regional processes affect invasion success in relation to ecosystem productivity and intensity of disturbances” Co-PI with A Dzialowski, \$54,000 \\
2009--2012 & National Science Foundation (NSF) “Collaborative Research: Characterizing the constraints on virus infection of cyanobacteria” Co-PI with S Wilhelm, \$500,000 \\
2008--2011 & Environmental Change Institute, University of Illinois, “Terrestrial carbon loss to aquatic ecosystems: pattern detection and hypothesis testing at the regional scale” Co-PI with J. Fraterrigo, \$25,000 \\
2009--2010 & Gordon \& Betty Moore Foundation and the Broad Institute, “Identifying viral mechanisms involved in rapid co-evolutionary dynamics between marine \textit{Synechococcus} and its phage” PI, payment in kind sequencing \\
2009--2012 & National Science Foundation (NSF) “Terrestrial carbon in aquatic ecosystems: experimental tests of the subsidy-stability hypothesis” PI, \$350,695 \\
2008--2011 & United States Department of Agriculture (USDA) “Moisture variability as a master regulator of microbial diversity and soil respiration across an agricultural landscape” PI, \$324,000 \\
2008--2010 & Center for Water Sciences, Michigan State University, “Microbial and ecosystem responses to land-water linkages: the energetic importance of terrestrial-derived dissolved organic carbon (DOC) in lakes” PI, \$142,318 \\
2007--2010 & National Science Foundation (NSF) “Water level fluctuations and internal eutrophication in lakes and wetland” Co-PI with S Hamilton, \$391,734 \\
2007--2010 & Michigan Agricultural Experiment Station, Rackham Foundation “Microbial responses to soil moisture variability in agricultural landscapes” PI, \$75,000 \\
2006--2009 & United States Department of Agriculture (USDA) “Pulsed ecosystem activity: Responses of soil microorganisms to variable water supply” PI, \$110,000 \\
2007 & Center for Water Sciences, Michigan State University “Towards a mechanistic framework of how changing temperatures affect aquatic bacterial community structure and function” PI, \$28,254 \\
2006--2009 & Center for Water Sciences, Michigan State University “Quantifying biogeochemical processes in flow-through wetlands” Co-PI with S Hamilton, \$145,036 \\
2005--2007 & Environmental Protection Agency (EPA) and the NH Department of Environmental Services “Using dispersal and environmental variables to predict the occurrence and susceptibility to invasion by non-native milfoil” Co-PI with R Thum, \$50,000 \\
2002--2004 & National Science Foundation (NSF) “Doctoral Dissertation Improvement Grant (DDIG): Linking lakes with the landscape: fate of terrestrial carbon in plankton food webs” Co-PI with K Cottingham, \$8,075 \\
2002--2004 & United States Geological Survey (USGS) \& National Institutes for Water Resources (NIWR) “Linking lakes with the landscape: fate of terrestrial carbon in planktonic food webs” PI, \$30,020 \\
\end{tabularx}

\section*{Teaching and mentoring}
\vspace{-0.5em}
Within my laboratory group, I have trained dozens of graduate students and postdocs who have gone on to have successful careers, not only in “R1” universities, but also smaller institutions, including tribal colleges. An equal number of trainees have secured careers in industry (e.g., agricultural genomics, health care informatics, food microbiology, synthetic biology) and non-profit organizations. In addition, I have mentored more than 50 undergraduate and high school students, as well as K-12 teachers. As mentioned in my teaching philosophy statement below, I am passionate about teaching both in a traditional classroom setting, in the computer laboratory, and at field stations and marine laboratories. I created and co-directed a hands-on summer course that trained dozens of students from North American and Europe in field-, molecular-, and informatic-based methods required for interdisciplinary Microbial Metagenomics. I reached an equally large and diverse groups of students as an Instructor for the MBL-based course in Microbial Diversity in Woods Hole where I exposed students to mathematical modeling and experimental evolution of marine bacteria and their viruses.


\section*{Service}
\vspace{-0.5em}
\textnormal{\underline{Science Advisor:}}\\[-2.5em]
\begin{longtable}{@{}p{3em}@{\hspace{3.5em}}p{0.87\textwidth}@{}}
2024--     & A Global Partnership to Address Climate \& Biodiversity Crisis, American Society for Microbiology (ASM) and the International Union of Microbiological Societies (IUMS) \\
2019--     & Ivy Tech, Biology Advisory Board \\
2015--2019 & Shedd Aquarium, Aquarium Microbiome Project, Chicago, Illinois, USA \\
\end{longtable}

\vspace{-0.5em}
\textnormal{\underline{Editor:}}\\[-2.5em]
\begin{longtable}{@{}p{3em}@{\hspace{3.5em}}p{0.87\textwidth}@{}}
2021--2027 & Editorial Board, \textit{ISME Journal} (\textit{International Society for Microbial Ecology}) \\
2016--2022 & Editor, \textit{Environmental Microbiology} and \textit{Environmental Microbiology Reports} \\
2010--2020 & Associate Editor, \textit{Frontiers in Terrestrial Microbiology} \\
\end{longtable}




\vspace{-0.5em}
\textnormal{\underline{Professional societies}} \\[-2.5em]
\begin{longtable}{@{}p{4em}@{\hspace{2em}}p{0.85\textwidth}@{}}
2025--2026 & Chair, Academy Leadership Nomination Subcommittee, American Academy of Microbiology \\
2024--2025 & Scientific Program Leader, American Society for Microbiology \\
2024--2025 & Finance Committee, Ecological Society of America \\
2023--2024 & Strategic Planning Working Group, Ecological Society of America \\
2023--2025 & Audit Committee, Ecological Society of America \\
2022--2024 & Publications Committee, Ecological Society of America \\
2022--2023 & Track Leader, Climate Change, American Society for Microbiology \\
2022--2025 & Chair, Academy Scientific Task Force (ASAT) on Climate Change and Microbiology, American Academy of Microbiology (AAM) \\
2021--2024 & Committee Member, Ecological Society of America (ESA), Fellows and Early Career Fellows \\
2021--2022 & Ad-hoc Program Evaluation Committee (APEC), American Society for Microbiology \\
2020--2021 & Ex officio, Program Committee, Ecology, Evolution, and Biodiversity (EEB); American Society for Microbiology \\
2018--2020 & Track Leader, Ecology, Evolution, and Biodiversity (EEB); American Society for Microbiology \\
2017--2019 & Member, Council on Microbial Sciences (COMS), American Society for Microbiology \\
2017--2020 & Program Committee, American Society for Microbiology, Representative for Ecological and Evolutionary Science \\
2017--2020 & Chair, Microbial Ecology (N) Division, American Society for Microbiology \\
2016--2018 & Member, American Society for Microbiology, Committee for K--12 Outreach \\
2015--2016 & Member, American Society for Microbiology, Communication Committee’s Environmental Microbiology Taskforce \\
2016 & Abstract Reviewer, American Society for Microbiology General Meeting, Ecological and Evolutionary Science Track \\
2010--2011 & Chair, Microbial Ecology Section, Ecological Society of America \\
2009--2010 & Vice Chair, Microbial Ecology Section, Ecological Society of America \\
2008--2009 & Secretary, Microbial Ecology Section, Ecological Society of America \\
2011 & Tom Frost Award Committee, Ecological Society of America \\
\end{longtable}


\vspace{-0.5em}
\textnormal{\underline{University and College}} \\[-2.5em]

\begin{longtable}{@{}p{4em}@{\hspace{2em}}p{0.85\textwidth}@{}}
2024--2027 & College Research Faculty Promotion Subcommittee \\
2024       & Search committee, Faculty 100 Initiative, Synthetic Biology \\
2021--2023 & Member of the College of Arts and Sciences Faculty IT Advisory Council \\
2021--2022 & Search committee, Soil Microbiologist, O’Neill School \\
2024       & Section Associate Chair (interim); Evolution, Ecology, and Behavior (EEB) \\
2020--2023 & Section Associate Chair; Evolution, Ecology, and Behavior (EEB) \\
2018--2019 & Section Associate Chair (interim); Evolution, Ecology, and Behavior (EEB) \\
2018       & Chair, Biology Graduate Admissions Committee \\
2017       & EEB Graduate Program Director (GPD) \\
2014--2020 & Advisory committee, Center for Genomics and Bioinformatics (CGB) \\
2013--2019 & Executive committee member, IU Research and Training Preserve (IURTP) \\
2013--2016 & Member, Departmental Planning Committee (DPC) \\
2014--2019 & Faculty advisor, Ecolunch \\
2015, 2016 & Member, Biology Graduate Admissions Committee \\
2012--2014 & Member, Biology Graduate Recruiting Weekend \\
2011       & Site representative, LTER Science Council meeting, Jekyll Island, Georgia, USA \\
2006--2009 & Executive board member, Biogeochemistry Environmental Research Initiative (BERI), Michigan State University \\
\end{longtable}

\end{document}
