\documentclass[11pt]{article}

% Match Word's default text area (6.5" x 9")
\usepackage[letterpaper, top=1in, bottom=1in, left=1in, right=1in]{geometry}

\usepackage{hyperref}
\usepackage{etaremune}
\usepackage{natbib}
\usepackage{url}
\usepackage{enumitem}
\setlist[itemize]{noitemsep, topsep=0pt}
\usepackage{parskip}   % Adds spacing between paragraphs, no indent
\linespread{1.3} % Slightly increase line spacing 
\usepackage[none]{hyphenat}
\usepackage{array}
\usepackage{tabularx} 
\setlength{\parindent}{0pt} % removes default indent
\usepackage{longtable}
\usepackage{ltablex}
\keepXColumns
\renewcommand{\refname}{Publications} % for citations
\renewcommand{\arraystretch}{1.3} % or 1.3 for more space between rows in tables
\usepackage[utf8]{inputenc}  % For pdflatex
\usepackage[T5]{fontenc} % For Vietnamese support


\begin{document}

% Centered header
\begin{center}
  {\LARGE \textbf{Jay-Terrence Lennon}}\\[0.5em]
  Department of Biology, Indiana University, Bloomington, Indiana 47405, USA \\
  %Phone: (812) 856-0962 \\
  Email: \href{mailto:lennonj@iu.edu}{lennonj@iu.edu} \\
  Lab website: \url{https://lennonlab.github.io} \\
  Lab wiki: \url{https://lennon.bio.indiana.edu} \\
  Google Scholar: \url{https://goo.gl/qx4hHR}
\end{center}

%\vspace{1em}

\section*{Education}
\noindent
\begin{tabular}{@{}l@{\hspace{3em}}l@{\hspace{3em}}l@{\hspace{3em}}l@{}}
1995 & B.S.    & Environmental Forest Biology     & SUNY-ESF at Syracuse \\
1999 & M.A.    & Ecology and Evolutionary Biology  & University of Kansas \\
2004 & Ph.D. & Ecology and Evolutionary Biology  & Dartmouth College \\
\end{tabular}


\section*{Professional Experience}
\vspace{-1.25em} % Adjust this value as needed 
\noindent
\begin{tabularx}{\textwidth}{@{}l@{\hspace{2em}}X@{}}
2023         & Visiting Professor, Goethe University, Frankfurt, Germany \\
2023         & Short-term Visiting Professor, ETH Zürich, Centre for Origin and Prevalence of Life \\
2020--2024   & Faculty, Complex Networks and Systems, Indiana University \\
2018--2022   & Faculty, Microbial Diversity Course, Marine Biological Laboratory, Woods Hole \\
2016--       & Professor, Indiana University, Department of Biology; Core Faculty in Evolution, Ecology \& Behavior (EEB); Affiliated Faculty in Microbiology \\
2016--2017   & Whitman Center Associate, Marine Biological Laboratory, Woods Hole \\
2016--2017   & Visiting Professor, Montana State University, Department of Microbiology and Immunology \\
2012--2016   & Associate Professor, Indiana University, Department of Biology; Core Faculty in EEB; Affiliated Faculty in Microbiology \\
2012--2015   & Adjunct Professor, W.K. Kellogg Biological Station, Michigan State University \\
2012         & Associate Professor, W.K. Kellogg Biological Station, Department of Microbiology and Molecular Genetics, Michigan State University \\
2011--       & Ad hoc Graduate Faculty, Michigan Technological University \\
2008--2012   & Adjunct Professor, Plant Biology Department, Michigan State University \\
2006--2012   & Assistant Professor, W.K. Kellogg Biological Station, Department of Microbiology and Molecular Genetics, Michigan State University \\
2004--2006   & Postdoctoral Research Associate, Brown University, Department of Ecology and Evolutionary Biology \\
\end{tabularx}

\section*{Honors and Awards}
\vspace{-1.25em} % Adjust this value as needed 
\noindent
\begin{tabularx}{\textwidth}{@{}l@{\hspace{2em}}X@{}}
2024        & Highly Cited Author, American Society for Microbiology (ASM) \\
2023        & Humboldt Prize, Alexander von Humboldt Foundation \\
2022--2025  & Governing Board, Ecological Society of America (ESA) \\
2022--2027  & Chair, Climate Change Task Force, American Academy of Microbiology (AAM) \\
2021        & Fellow, Ecological Society of America (ESA) \\
2020--2022  & Distinguished Lecturer, American Society for Microbiology (ASMDL) \\
2020--2026  & Governor, American Academy of Microbiology (AAM) \\
2019        & Fellow, American Academy of Microbiology (AAM) \\
2019--2024  & Highly Cited Researcher, Clarivate, Cross-Field \\
2018        & Fellow, American Association for the Advancement of Science (AAAS) \\
2012        & Kavli Fellow, National Academy of Sciences \\
2004        & USDA National Research Initiative (NRI) Postdoctoral Fellowship Award \\
2004        & Hannah T. Croasdale Graduate Scholar Award, Dartmouth College \\
2004        & Milton L. Shifman Endowed Scholarship, Marine Biological Laboratory \\
2004        & Albert Cass Fellowship, The Rockefeller University \\
2004        & Nathan Jenks Biology Award, Dartmouth College \\
2003        & Best student presentation, North American Lake Management Society National Meeting, Mashantucket, CT \\
2002        & NSF Doctoral Dissertation Improvement Grant (DDIG) \\
1999--2004  & Dartmouth Fellowship, Dartmouth College \\
1995        & Undergraduate honors: \textit{Magna Cum Laude}; President’s List; Alpha Sigma Xi, SUNY-ESF \\
1992        & Outstanding history student, SUNY Oswego \\
\end{tabularx}

\section*{Publications}
\vspace{-0.25em} % Adjust this value as needed 
\begin{etaremune}

\item[] \textnormal{\underline{Preprints:}}

\item Măgălie A, Marantos A, Schwartz DA, Marchi J, Lennon JT, Weitz JS (2024) Phage infection fronts trigger early sporulation and collective defense in bacterial populations. \textit{bioRxiv}. doi:10.1101/2024.05.22.595388. \href{https://www.biorxiv.org/content/10.1101/2024.05.22.595388v1.full.pdf}{(link)}

\item Hill CA, McMullen JC, Lennon JT (2024) Nitrogen enrichment alters selection on rhizobial genes. \textit{bioRxiv}. doi:10.1101/2024.11.25.625319. \href{https://www.biorxiv.org/content/10.1101/2024.11.25.625319v1.full.pdf}{(link)}

\item Hu A, Cui Y, Bercovici A, Tanentzap AJ, Lennon JT, Lin X, Yang Y, Liu Y, Osterholz H, Dong H, Lu Y, Jiao N, Wang J (2024) Photochemical processes drive thermal responses of dissolved organic matter in the dark ocean. \textit{bioRxiv}. doi:10.1101/2024.09.06.611638. \href{https://www.biorxiv.org/content/10.1101/2024.09.06.611638v1.full.pdf}{(link)}

\item McGill B, Jarzyna M, Diaz R, Barnes C, Diaz FH, Economo E, French C, Hagen O, James H, Kivlin S, Lahiri S, Lennon JT, Mascarenhas R, Ohyama L, Rabosky DL, Zhu K, Hickerson M, Gillespie R (In review) A call to develop a coherent discipline of biodiversity science to address global change.

\vspace{1em}
\item[] \textnormal{\underline{Patents:}}
\item Lennon JT, van der Elst LA, Mueller EA, Gumennik A (2023) Gut bioreactor and method for making the same. US Patent 11,840,681 B2. \href{https://lennonlab.github.io/assets/publications/Lennon_etal_2023b.pdf}{(pdf)}

\vspace{1em}
\item[] \textnormal{\underline{White papers:}}

\item Rappuoli R, Nguyen N, Bloom DE, Brooks CG, Egamberdieva D, Lawley TD, Morhard R, Mukhopadhyay A, Lennon JT, Peixoto RS, Silver PA, Stein LY (2025) Microbial solutions for climate change — Toward an economically resilient future. \textit{American Society for Microbiology}. \href{https://lennonlab.github.io/assets/publications/Rappuoli_etal_2025b.pdf}{(pdf)}

\item Lennon JT and 28 others (2023) Colloquium report: Microbes in models: integrating microbes into earth system models for understanding climate change. American Society for Microbiology, Washington, DC. \href{https://www.ncbi.nlm.nih.gov/books/NBK592518/}{(link)}

\item Lennon JT and 28 others (2023) Colloquium report: The role of microbes in mediating methane emissions. American Society for Microbiology, Washington, DC. \href{https://pubmed.ncbi.nlm.nih.gov/38194471/}{(link)}

\vspace{1em}
\item[] \textnormal{\underline{Commentaries and essays:}}

\item Lennon JT, Rappuoli R, Bloom DE, Brooks CG, Egamberdieva D, Lawley TD, Morhard R, Mukhopadhyay A, Nguyen N, Peixoto RS, Silver PA, Stein LY (2025) Microbial solutions for climate change require global partnership. \textit{mBio} 16: 10.1128/mbio.00778-25. \href{https://lennonlab.github.io/assets/publications/Lennon_etal_2025a.pdf}{(pdf)}

\item Rappuoli R, Nguyen N, Bloom DE, Brooks CG, Egamberdieva D, Lawley TD, Morhard R, Mukhopadhyay A, Lennon JT, Peixoto RS, Silver PA, Stein LY (2025) Microbes could help address climate change — why aren’t we using them? \textit{Nature} 639: 864–866. \href{https://lennonlab.github.io/assets/publications/Rappuoli_etal_2025a.pdf}{(pdf)}

\item Peixoto R, Voolstra CR, Stein LY, Hugenholtz P, Salles JF, Amin SA, Häggblom M, Gregory A, Makhalanyane TP, Wang F, Agbodjato NA, Wang Y, Jiao N, Lennon JT, Ventosa A, Bavoil PM, Miller V, Gilbert JA (2024) Microbiology at the brink: a unified call for action against climate catastrophe. 

\textit{Published in:}
\begin{itemize}
  \item Nature Microbiology 9: 3084–3085 \href{https://lennonlab.github.io/assets/publications/Peixoto_etal_2024a.pdf}{(pdf)}
  \item Nature Communications 15: 9637 \href{https://lennonlab.github.io/assets/publications/Peixoto_etal_2024b.pdf}{(pdf)}
  \item Nature Reviews Microbiology 23: 1–2 \href{https://lennonlab.github.io/assets/publications/Peixoto_etal_2024c.pdf}{(pdf)}
  \item Nature Reviews Earth and Environment 6: 4–5 \href{https://lennonlab.github.io/assets/publications/Peixoto_etal_2024d.pdf}{(pdf)}
  \item ISMEJ 18: wrae219 \href{https://lennonlab.github.io/assets/publications/Peixoto_etal_2024e.pdf}{(pdf)}
  \item mSystems 11: e0141624 \href{https://lennonlab.github.io/assets/publications/Peixoto_etal_2024f.pdf}{(pdf)}
  \item Communications Biology 7: 1466 \href{https://lennonlab.github.io/assets/publications/Peixoto_etal_2024g.pdf}{(pdf)}
  \item Communications Earth and Environment 5: 672 \href{https://lennonlab.github.io/assets/publications/Peixoto_etal_2024h.pdf}{(pdf)}
  \item FEMS Microbiology Ecology 100: fiae144 \href{https://lennonlab.github.io/assets/publications/Peixoto_etal_2024i.pdf}{(pdf)}
  \item NPJ Biodiversity 3: 34 \href{https://lennonlab.github.io/assets/publications/Peixoto_etal_2024j.pdf}{(pdf)}
  \item NPJ Biofilms and Microbiomes 10: 122 \href{https://lennonlab.github.io/assets/publications/Peixoto_etal_2024k.pdf}{(pdf)}
  \item NPJ Sustainable Agriculture 2: 23 \href{https://lennonlab.github.io/assets/publications/Peixoto_etal_2024l.pdf}{(pdf)}
  \item NPJ Climate Action 3: 1–3 \href{https://lennonlab.github.io/assets/publications/Peixoto_etal_2024m.pdf}{(pdf)}
  \item Sustainable Microbiology 1: qvae029 \href{https://lennonlab.github.io/assets/publications/Peixoto_etal_2024n.pdf}{(pdf)}
\end{itemize}

\item Beattie GA, Cotrufo FM, Crowther TW, Edlund A, Salles JF, Gilbert JK, Jansson JK, Jensen PR, Lennon JT, Makhalanyane T, Martiny JBH, Newman DK, Stevenson M (2024) Soil microbial strategies for climate mitigation: Report from a climate action workshop in Las Vegas, Nevada, February 2024. \textit{Sustainable Microbiology} 1: qvae033. \href{https://lennonlab.github.io/assets/publications/Beattie_etal_2024.pdf}{(pdf)}

\item Lennon JT (2020) Microbial life underfoot. \textit{mBio} 11: e03201-19. \href{https://lennonlab.github.io/assets/publications/Lennon_2020.pdf}{[pdf]}

\item Lennon JT, Locey KJ (2018) There are more microbial species on Earth than stars in the galaxy. \textit{Aeon}. \href{https://aeon.co/ideas/there-are-more-microbial-species-on-earth-than-stars-in-the-sky}{(link)}

\item Lau JA, Lennon JT, Heath KD (2017) Trees harness the power of microbes to survive climate change. \textit{Proceedings of the National Academy of Sciences of the United States of America} 114: 11009–11011. \href{https://lennonlab.github.io/assets/publications/Lau_etal_2017.pdf}{(pdf)}

\vspace{1em}
\item[] \textnormal{\underline{Book reviews:}}

\item Wisnoski NI, Lennon JT (2016) Book Review. Principles of Microbial Diversity by James W. Brown. \textit{Quarterly Review of Biology} 91: 98–99. \href{https://lennonlab.github.io/assets/publications/Wisnoski_Lennon_2016.pdf}{(pdf)}

\item Moger-Reischer RZ, Lennon JT (2017) Book Review. \textit{The human superorganism: how the microbiome is revolutionizing the pursuit of a healthy life} by Rodney Dietert. \textit{Quarterly Review of Biology} 92: 203. \href{https://lennonlab.github.io/assets/publications/Moger-Reicher_Lennon_2017.pdf}{(pdf)}

\vspace{1em}
\item[] \textnormal{\underline{Theses:}}

\item Lennon JT (1999) Invasion success of the exotic \textit{Daphnia lumholtzi}: species traits and community resistance. University of Kansas, 77 pp.

\item Lennon JT (2004) The energetic importance of terrestrial carbon in lake ecosystems. Dartmouth College, 169 pp. \href{https://lennonlab.github.io/assets/publications/Lennon_2004_Thesis.pdf}{(pdf)}

\vspace{1em}
\item[] \textnormal{\underline{Peer-reviewed papers:}}

\item Lennon JT, Lehmkuhl BK, Chen L, Illingworth M, Kuo V, Muscarella ME (2025) \\Resuscitation-promoting factor (Rpf) terminates dormancy among diverse soil bacteria.\\
\textit{mSystems} 10: 10.1128/msystems.01517-24. \href{https://lennonlab.github.io/assets/publications/Lennon_etal_2025b.pdf}{(pdf)}

\item Mueller EA, Lennon JT (2024) Residence time structures microbial communities through niche partitioning. \textit{Ecology Letters} 28: e70093. \href{https://lennonlab.github.io/assets/publications/Mueller_Lennon_2025.pdf}{(pdf)}

\item Nevermann HD, Gros C, Lennon JT (2024) A game of life with dormancy. \textit{Proceedings of the Royal Society B} 292: 20242543. \href{https://lennonlab.github.io/assets/publications/Nevermann_etal_2025.pdf}{(pdf)}

\item Zang Z, Zhang C, Park KJ, Schwartz DA, Podicheti R, Lennon JT, Gerdt JP (2025) Streptomyces secretes a siderophore that sensitizes competitor bacteria to phage infection. \textit{Nature Microbiology}. \href{https://lennonlab.github.io/assets/publications/Zang_etal_2025.pdf}{(pdf)}

\item Beattie GA, Edlund A, Esiobu N, Gilbert J, Nicolaisen MH, Jansson JK, Jensen P, Keiluwei M, Lennon JT, Martiny JBH, Minnisi VR, Newmann D, Peixoto R, Schadt C, van der Meer JR (2025) Soil microbiome interventions for carbon sequestration and climate mitigation. \textit{mSystems}. \href{https://lennonlab.github.io/assets/publications/Beattie_etal_2025.pdf}{(pdf)}

\item Webster KD, Lennon JT (2025) Dormancy in the origin, evolution, and persistence of life on Earth. \textit{Proceedings of the Royal Society B: Biological Sciences} 292: 20242035. \href{https://lennonlab.github.io/assets/publications/Webster_Lennon_2025.pdf}{(pdf)}

\item Waldrop MP, Ernakovich JG, Vishnivetskaya TA, Schaefer SR, Mackleprang R, Bara J, O'Brien JM, Winkel M, Barbato RA, Heffernan L, Leewis MC, Hewitt RE, Hultman J, Sun Y, Biasi C, Bradley JA, Liebner S, Ricketts MP, Muscarella ME, Schütte U, Abuah F, Whalen E, Timling I, Voight C, Taş N, Lloyd KG, Silganen HMP, Rivkina EM, Voříšková J, Tao J, Liang R, Lennon JT, Onstott TC (2025) Microbial ecology of permafrost soils: populations, processes, and perspectives. \textit{Permafrost and Periglacial Processes}. \href{https://lennonlab.github.io/assets/publications/Waldrop_etal_2025.pdf}{(pdf)}

\item Wu W, Hsieh C, Logares R, Lennon JT, Liu H (2024) Ecological processes shaping highly connected bacterial communities along strong environmental gradients. \textit{FEMS Microbiology Ecology} 100: fiae146. \href{https://lennonlab.github.io/assets/publications/Wu_etal_2024.pdf}{(pdf)}

\item Hu A, Jang KS, Tanentzap AJ, Zhao W, Lennon JT, Liu J, Li M, Stegen JC, Choi M, Lu Y, Feng X, Wang J (2024) Thermal responses of dissolved organic matter under global change. \textit{Nature Communications} 15: 576. \href{https://lennonlab.github.io/assets/publications/Hu_etal_2024.pdf}{(pdf)}

\item Lennon JT, Abramoff RZ, Allison SD, Burckhardt RM, DeAngelis KM, Dunne JP, Frey SD, Friedlingstein P, Hawkes CV, Hungate BA, Khurana S, Kivlin SN, Levine N, Manzoni S, Martiny AC, Martiny JBH, Nguyen N, Rawat M, Talmy D, Todd-Browne K, Vogt M, Wieder WR, Zakem E (2024) Priorities, opportunities, and challenges for integrating microorganisms into Earth system models for climate change prediction. \textit{mBio} 15: e00455-24. \href{https://lennonlab.github.io/assets/publications/Lennon_etal_2024.pdf}{(pdf)}

\item Moger-Reischer RZ, Glass JI, Wise KS, Sun L, Bittencourt DMC, Lehmkuhl BK, Schoolmaster DR Jr, Lynch M, Lennon JT (2023) Evolution of a minimal cell. \textit{Nature} 620: 122--127. \href{https://lennonlab.github.io/assets/publications/Moger-Reischer_etal_2023.pdf}{(pdf)}

\item Schwartz DA, Shoemaker WR, Măgăliee A, Weitz JS, Lennon JT (2023) Bacteria-phage coevolution with a seed bank. \textit{ISMEJ} 17: 1315–1325. \href{https://lennonlab.github.io/assets/publications/Schwartz_etal_2023b.pdf}{(pdf)}

\item Fishman FJ, Lennon JT (2023) Macroevolutionary constraints on global microbial diversity. \textit{Ecology and Evolution} 13: e10403. \href{https://lennonlab.github.io/assets/publications/Fishman_Lennon_2023.pdf}{(pdf)}

\item Zhou X, Lennon JT, Lu X, Ruan A (2023) Anthropogenic activities mediate stratification and stability of microbial communities in freshwater sediments. \textit{Microbiome} 11: 191. \href{https://lennonlab.github.io/assets/publications/Zhou_etal_2023.pdf}{(pdf)}

\item Schwartz DA, Rodriguez-Ramos J, Shaffer M, Flynn F, Daly R, Wrighton KC, Lennon JT (2023) Human-gut phages harbor sporulation genes. \textit{mBio} e0018223. \href{https://lennonlab.github.io/assets/publications/Schwartz_etal_2023a.pdf}{(pdf)}

\item Wisnoski NI, Lennon JT (2023) Scaling up and down: movement ecology for microorganisms. \textit{Trends in Microbiology} 31: 242–253. \href{https://lennonlab.github.io/assets/publications/Wisnoski_Lennon_2023.pdf}{(pdf)}

\item Măgălie A, Schwartz DA, Lennon JT, Weitz JS (2023) Optimal dormancy strategies in fluctuating environments given delays in phenotypic switching. \textit{Journal of Theoretical Biology} 561: 111413. \href{https://lennonlab.github.io/assets/publications/Magalie_etal_2023.pdf}{(pdf)}

\item Lennon JT, Frost SDW, Nguyen NK, Peralta AL, Place AR, Treseder KK (2023) Microbiology and climate change: a transdisciplinary imperative. \textit{mBio} 13: e0335-22. \href{https://lennonlab.github.io/assets/publications/Lennon_etal_2023.pdf}{(pdf)}

\item Bolin LG, Lennon JT, Lau JA (2023) Traits of soil bacteria predict plant responses to soil moisture. \textit{Ecology} 104: e3893. \href{https://lennonlab.github.io/assets/publications/Bolin_etal_2023.pdf}{(pdf)}

\item Irvine R, Houser M, Marquart-Pyatt ST, Bolin L, Browning EG, Dott G, Evans SE, Howard M, Lau J, Lennon JT (2023) Soil health through farmers’ eyes: Toward a better understanding of how farmers view, value, and manage for healthier soils. \textit{Journal of Soil and Water Conservation} 78: 82–92. \href{https://lennonlab.github.io/assets/publications/Irvine_etal_2023.pdf}{(pdf)}

\item McMullen JG, Lennon JT (2023) Mark-recapture of microorganisms.\textit{Environmental \\ Microbiology} 25: 150–157. \href{https://lennonlab.github.io/assets/publications/McMullen_Lennon_2023.pdf}{(pdf)}

\item Webster KD, Schimmelmann A, Drobniak A, Mastalerz M, Lagarde LR, Boston PJ, Lennon JT (2022) Diversity and composition of cave methanotrophic communities. \textit{Microbiology Spectrum} 10: e0156621. \href{https://lennonlab.github.io/assets/publications/Webster_etal_2022.pdf}{(pdf)}

\item Schwartz DA, Lekmkuhl BK, Lennon JT (2022) Phage-encoded sigma factors alter bacterial dormancy. \textit{mSphere} e00927-22. \href{https://lennonlab.github.io/assets/publications/Schwartz_etal_2022.pdf}{(pdf)}

\item Shoemaker WR, Polezhaeva E, Givens KB, Lennon JT (2022) Seed banks alter the molecular evolutionary dynamics of \textit{Bacillus subtilis}. \textit{Genetics} 221: iyac071. \href{https://lennonlab.github.io/assets/publications/Shoemaker_etal_2022.pdf}{(pdf)}

\item Hu A, Choi M, Tanentzap AJ, Liu J, Jang KS, Lennon JT, Liu Y, Soininen J, Lu X, Zhang Y, Shen J, Wang J (2022) Ecological networks of dissolved organic matter and microorganisms under global change. \textit{Nature Communications} 13: 3699. \href{https://lennonlab.github.io/assets/publications/Hu_etal_2022a.pdf}{(pdf)}

\item Hu A, Jang KS, Meng F, Stegen J, Tanentzap AJ, Choi M, Lennon JT, Soinenen J, Wang J (2022) Microbial and environmental processes shape the link between organic matter functional traits and composition. \textit{Environmental Science and Technology} 56: 10504-10516. \href{https://lennonlab.github.io/assets/publications/Hu_etal_2022b.pdf}{(pdf)}

\item Krause SMB, Bertilsson S, Grossart HP, Bodelier PLE, van Bodegom P, Lennon JT, Philippot L, Le Roux X (2022) Microbial trait-based approaches for agroecosystems. \textit{Advances in Agronomy} 175: 260–299. \href{https://lennonlab.github.io/assets/publications/Krause_etal_2022.pdf}{(pdf)}

\item Shoemaker WR Lennon JT (2022) Predicting parallelism and quantifying divergence in experimental evolution. \textit{mSphere} 7: e00672-21. \href{https://lennonlab.github.io/assets/publications/Shoemaker_Lennon_2022.pdf}{(pdf)}

\item Shoemaker WR Jones SE Muscarella ME Behringer MG Lehmkuhl BK Lennon JT (2021) Microbial population dynamics and evolutionary outcomes under extreme energy-limitation. \textit{Proceedings of the National Academy of Sciences of the United States of America} 118: e2101691118. \href{https://lennonlab.github.io/assets/publications/Shoemaker_etal_2021b.pdf}{(pdf}, \href{https://lennonlab.github.io/assets/publications/Rillig_etal_2021.pdf}{commentary}, \href{https://lennonlab.github.io/assets/publications/Shoemaker_etal_2021b_Suppl.pdf}{supplement)}

\item Lennon JT den Hollander F Wilke-Berenguer M Blath J (2021) Principles of seed banks: complexity emerging from dormancy. \textit{Nature Communications} 2: 4807. \href{https://lennonlab.github.io/assets/publications/Lennon_etal_2021a.pdf}{(pdf)}

\item Wisnoski NI Lennon JT (2021) Stabilising role of seed banks and the maintenance of bacterial diversity. \textit{Ecology Letters} 24: 2328–2338. \href{https://lennonlab.github.io/assets/publications/Wisnoski_Lennon_2021.pdf}{(pdf}, \href{https://lennonlab.github.io/assets/publications/Wisnoski_Lennon_2021_Suppl.pdf}{supplement)}

\item Shoemaker WR Polezhaeva E Givens KB Lennon JT (2021) Molecular evolutionary dynamics of energy limited microorganisms. \textit{Molecular Biology and Evolution} 38: msab195. \href{https://lennonlab.github.io/assets/publications/Shoemaker_etal_2021a.pdf}{(pdf}, \href{https://lennonlab.github.io/assets/publications/Shoemaker_etal_2021a_Suppl.zip}{supplement)}

\item Lamit LJ, Romanowicz KJ, Potvin LR, Lennon JT, Tringe SG, Chimner RA, Kolka RK, Kane ES, Lilleskov EA (2021) Peatland microbial community responses to plant functional group and drought are depth-dependent. \textit{Molecular Ecology} 30: 5119–5136. \href{https://lennonlab.github.io/assets/publications/Lamit_etal_2021.pdf}{[pdf]}

\item Kuo V, Lehmkuhl BK, Lennon JT (2021) Resuscitation of the microbial seed bank alters plant-soil interactions. \textit{Molecular Ecology} 30: 2905–2914. \href{https://lennonlab.github.io/assets/publications/Kuo_etal_2021.pdf}{[pdf]}

\item Wisnoski NI, Lennon JT (2020) Microbial community assembly in a multi-layer dendritic metacommunity. \textit{Oecologia} 195: 13–24. \href{https://lennonlab.github.io/assets/publications/Wisnoski_Lennon_2020.pdf}{[pdf]}

\item Mobilian C, Wisnoski NI, Lennon JT, Abler M, Widney S, Craft CB (2020) Differential effects of press vs. pulse seawater intrusion on microbial communities of a tidal freshwater marsh. \textit{Limnology and Oceanography Letters} 8: 154–161. \href{https://lennonlab.github.io/assets/publications/Mobilian_etal_2020.pdf}{[pdf]}

\item Muscarella ME, Howey XM, Lennon JT (2020) Trait-based approach to bacterial growth efficiency. \textit{Environmental Microbiology} 22: 3494–3504. \href{https://lennonlab.github.io/assets/publications/Muscarella_etal_2020.pdf}{[pdf]}

\item Moger-Reischer RZ, Snider EZ, McKenzie KL, Lennon JT (2020) Low costs of adaptation to dietary restriction. \textit{Biology Letters} 16: 20200008. \href{https://lennonlab.github.io/assets/publications/Moger-Reischer_etal_2020.pdf}{[pdf]}

\item Locey KJ, Muscarella ME, Larsen ML, Bray SR, Jones SE, Lennon JT (2020) Dormancy dampens the microbial distance-decay relationship. \textit{Philosophical Transactions of the Royal Society B} 375: 20190243. \href{https://lennonlab.github.io/assets/publications/Locey_etal_2020.pdf}{[pdf]}

\item Lennon JT, Locey KJ (2020) More evidence for Earth's massive microbiome. \textit{Biology Direct} 15: 5. \href{https://lennonlab.github.io/assets/publications/Lennon_Locey_2020.pdf}{[pdf]}

\item Wisnoski NI, Muscarella ME, Larsen ML, Peralta AP, Lennon JT (2020) Metabolic insight into bacterial community assembly across ecosystem boundaries. \textit{Ecology} 101: e02968. \href{https://lennonlab.github.io/assets/publications/Wisnoski_etal_2020.pdf}{[pdf]}

\item Yin Y, Masalerz M, Lennon JT, Drobniak A, Schimmelmann A (2020) Characterization and microbial mitigation of fugitive methane emissions from oil and gas wells: Example from Indiana, USA. \textit{Applied Geochemistry} 118: 104619. \href{https://lennonlab.github.io/assets/publications/Yin_etal_2020.pdf}{(pdf)}

\item Mueller EA, Wisnoski NI, Peralta AL, Lennon JT (2019) Microbial rescue effects: how microbiomes can save hosts from extinction. \textit{Functional Ecology} 34: 2055--2064. \href{https://lennonlab.github.io/assets/publications/Mueller_etal_2019.pdf}{(pdf)}

\item Moger-Reischer RZ, Lennon JT (2019) Microbial aging and longevity. \textit{Nature Reviews Microbiology} 17: 79--690. \href{https://lennonlab.github.io/assets/publications/Moger-Reischer_Lennon_2019.pdf}{(pdf)}

\item Wisnoski NI, Leibold MA, Lennon JT (2019) Dormancy in metacommunities. \textit{American Naturalist} 194: 131--151. \href{https://lennonlab.github.io/assets/publications/Wisnoski_etal_2019.pdf}{(pdf)}

\item Salazar A, Lennon JT, Dukes JS (2019) Microbial activity improves predictability of soil respiration dynamics. \textit{Biogeochemistry} 144: 103--116. \href{https://lennonlab.github.io/assets/publications/Salazar_etal_2019.pdf}{(pdf)}

\item Muscarella ME, Boot CM, Broeckling CD, Lennon JT (2019) Resource diversity structures aquatic bacterial communities. \textit{ISMEJ} 13: 2183--2195. \href{https://lennonlab.github.io/assets/publications/Muscarella_etal_2019.pdf}{(pdf)}

\item Larsen ML, Wilhelm SW, Lennon JT (2019) Nutrient stoichiometry shapes microbial coevolution. \textit{Ecology Letters} 22: 1009--1018. \href{https://lennonlab.github.io/assets/publications/Larsen_etal_2019.pdf}{(pdf)}

\item Locey KJ, Lennon JT (2019) A residence-time framework for biodiversity. \textit{American Naturalist} 194: 59--72. \href{https://lennonlab.github.io/assets/publications/Locey_Lennon_2019.pdf}{(pdf)}

\item Sprunger CD, Culman SW, Peralta AP, DuPont ST, Lennon JT, Snapp SS (2019) Perennial grain crop roots and nitrogen management shape soil food webs and soil carbon dynamics. \textit{Soil Biology and Biochemistry} 137: 107573. \href{https://lennonlab.github.io/assets/publications/Sprunger_etal_2019.pdf}{(pdf)}

\item Shade A, Dunn RR, Blowes SA, Keil P, Bohannan BMJ, Hermann M, Küsel K, Lennon JT, Sanders NJ, Storch D, Chase J (2018) Macroecology to unite all biodiversity great and small. \textit{Trends in Ecology and Evolution} 33: 731--744. \href{https://lennonlab.github.io/assets/publications/Shade_etal_2018.pdf}{(pdf)}

\item Lennon JT, Muscarella ME, Placella SA, Lehmkuhl BK (2018) How, when, and where relic DNA biases estimates of microbial diversity. \textit{mBio} 9: e00637-18. \href{https://lennonlab.github.io/assets/publications/Lennon_etal_2018.pdf}{(pdf)}

\item Shoemaker WR, Lennon JT (2018) Evolution with a seed bank: the population genetic consequences of microbial dormancy. \textit{Evolutionary Applications} 11: 60–75. \href{https://lennonlab.github.io/assets/publications/Shoemaker_Lennon_2018.pdf}{(pdf)}

\item Hall EK, Bernhardt ES, Bier R, Bradford MA, Boot CM, Cotner JB, del Giorgio PA, Evans SE, Graham EB, Jones SE, Lennon JT, Nemergut D, Osborne B, Rocca JD, Schimel JS, Waldrop MS, Wallenstein MW (2018) Understanding how microbiomes influence the systems they inhabit. \textit{Nature Microbiology} 3: 977–982. \href{https://lennonlab.github.io/assets/publications/Hall_etal_2018.pdf}{(pdf)}

\item Peralta AL, Sun Y, McDaniel MD, Lennon JT (2018) Crop diversity increases disease suppressive capacity of soil microbiomes. \textit{Ecosphere} 9: e02235. \href{https://lennonlab.github.io/assets/publications/Peralta_etal_2018.pdf}{(pdf)}

\item Long H, Sung W, Kucukyildirim S, Williams E, Miller S, Guo W, Patterson C, Gregory C, Strauss C, Stone C, Berne C, Kysela D, Shoemaker WR, Muscarella M, Luo H, Lennon JT, Brun YV, Lynch M (2018) Evolutionary determinants of genome-wide nucleotide composition. \textit{Nature Ecology and Evolution} 2: 237--240. \href{https://lennonlab.github.io/assets/publications/Long_etal_2018.pdf}{(pdf)}

\item Schimmelmann A, Streil T, Fernandez-Cortes A, Cuezva S, Lennon JT (2018) Radiolysis via radioactivity is not responsible for rapid methane oxidation in subterranean air. \textit{PLOS ONE} 113: 020650. \href{https://lennonlab.github.io/assets/publications/Schimmelmann_etal_2018.pdf}{(pdf)}

\item Shoemaker WR, Locey KJ, Lennon JT (2017) A macroecological theory of microbial biodiversity. \textit{Nature Ecology and Evolution} 1: 0107. \href{https://lennonlab.github.io/assets/publications/Shoemaker_etal_2017.pdf}{(pdf)}

\item Kuo V, Shoemaker WR, Muscarella ME, Lennon JT (2017) Whole genome sequence of the soil bacterium \textit{Micrococcus} sp. KBS0714. \textit{Genome Announcements} 5: e00697-17. \href{https://lennonlab.github.io/assets/publications/Kuo_etal_2017.pdf}{(pdf)}

\item Nguyễn-Thuỳ D, Schimmelmann A, Nguyễn-Văn H, Drobniak A, Lennon JT, Tạ PH, Nguyễn NTA (2017) Subterranean microbial oxidation of atmospheric methane in cavernous tropical karst. \textit{Chemical Geology} 466: 229--238. \href{https://lennonlab.github.io/assets/publications/Nguyen-Thuy_etal_2017.pdf}{(pdf)}

\item Lamit LJ, Romanowicz JH, Potvin LR, Rivers A, Singh K, Lennon JT, Tringe S, Kane E, Lilleskov E (2017) Patterns and drivers of fungal community depth stratification in \textit{Sphagnum} peat. \textit{FEMS Microbiology Ecology} 93: fix082. \href{https://lennonlab.github.io/assets/publications/Lamit_etal_2017.pdf}{(pdf)}

\item Locey KJ, Fisk MC, Lennon JT (2017) Microscale insight into microbial seed banks. \textit{Frontiers in Microbiology} 7: 2040. \href{https://lennonlab.github.io/assets/publications/Locey_etal_2017.pdf}{(pdf)}

\item Webster KD, Lagarde LR, Sauer PE, Schimmelmann A, Lennon JT, Boston PJ (2017) Isotopic evidence for the migration of thermogenic methane in Cueva de Villa Luz cave, Tabasco, Mexico. \textit{Journal of Cave and Karst Studies} 79: 24–34. \href{https://lennonlab.github.io/assets/publications/Webster_etal_2017.pdf}{(pdf)}

\item Lennon JT, Locey KJ (2017) Macroecology for microbiology. \textit{Environmental Microbiology Reports} 9: 38–40. \href{https://lennonlab.github.io/assets/publications/Lennon_Locey_2017.pdf}{(pdf)}

\item LaSarre B, McCully AL, Lennon JT, McKinlay JB (2017) Microbial mutualism dynamics governed by dose-dependent toxicity and growth-independent production of a cross-fed nutrient. \textit{ISMEJ} 11: 337–348. \href{https://lennonlab.github.io/assets/publications/LaSarre_etal_2017.pdf}{(pdf)}

\item Lennon JT, Nguyễn Thùy D, Phạm \DJ{}ức N, Drobniak A, Tạ PH, Phạm N\DJ{}, Streil T, Webster KD, Schimmelmann A (2017) Microbial contributions to subterranean methane sinks. \textit{Geobiology} 15: 254--258. \href{https://lennonlab.github.io/assets/publications/Lennon_etal_2017.pdf}{(pdf)}

\item Skelton J, Geyer KM, Lennon JT, Creed RP, Brown BL (2017) Multi-scale ecological filters shape crayfish microbiome assembly. \textit{Symbiosis} 72: 159–170. \href{https://lennonlab.github.io/assets/publications/Skelton_etal_2017.pdf}{(pdf)}

\item Locey KJ, Lennon JT (2017) A modeling platform for the simultaneous emergence of ecological patterns. \textit{PeerJ Preprints} 5: e1469v3. \href{https://lennonlab.github.io/assets/publications/Locey_Lennon_2017.pdf}{(pdf)}

\item Locey KJ, Lennon JT (2016) Powerful predictions of biodiversity from ecological models and scaling laws. \textit{Proceedings of the National Academy of Sciences of the United States of America} 113: E5097. \href{https://lennonlab.github.io/assets/publications/Locey_Lennon_2016_Reply.pdf}{(pdf)}

\item Lennon JT, Locey KJ (2016) The underestimation of global microbial diversity. \textit{mBio} 7: e01298-16. \href{https://lennonlab.github.io/assets/publications/Lennon_Locey_2016_mBio.pdf}{(pdf)}

\item Locey KJ, Lennon JT (2016) Scaling laws predict global microbial diversity. \textit{Proceedings of the National Academy of Sciences of the United States of America} 113: 5970–5975. \href{https://lennonlab.github.io/assets/publications/Locey_Lennon_2016.pdf}{(pdf}, \href{https://lennonlab.github.io/assets/publications/Locey_Lennon_2016_SI.pdf}{supplement}, \href{https://lennonlab.github.io/assets/publications/Pedros-Alio_Manrubia_2016.pdf}{commentary}, \href{https://f1000.com/prime/726327633}{F1000 recommendation)}

\item Lennon JT, Lehmkuhl BK (2016) A trait-based approach to biofilms in soil. \textit{Environmental Microbiology} 18: 2732–2742. \href{https://lennonlab.github.io/assets/publications/Lennon_Lehmkuhl_2016.pdf}{(pdf)}

\item Muscarella ME, Jones SE, Lennon JT (2016) Species sorting along a subsidy gradient alters community stability. \textit{Ecology} 97: 2034–2043. \href{https://lennonlab.github.io/assets/publications/Muscarella_etal_2016.pdf}{(pdf}, \href{https://lennonlab.github.io/assets/publications/Muscarella_etal_Supplement_2016.pdf}{supplement)}

\item Aanderud ZT, Vert JC, Magnusson TW, Lennon JT, Breakwell DP, Harker AR (2016) Bacterial dormancy is more prevalent in freshwater than hypersaline lakes. \textit{Frontiers in Microbiology} 7: 853. \href{https://lennonlab.github.io/assets/publications/Aanderud_etal_2016.pdf}{(pdf)}

\item Lennon JT, Denef VJ (2016) Evolutionary ecology of microorganisms: from the tamed to the wild. In: Yates MV, Nakatsu C, Miller R, Pillai S (eds). \textit{Manual of Environmental Microbiology, 4th ed.} ASM Press, Washington, DC, pp. 4.1.2-1–4.1.2-12. \href{https://lennonlab.github.io/assets/publications/Lennon_Denef_2015.pdf}{(pdf)}

\item Wigington CH, Sonderegger DL, Brussard CPD, Buchan A, Finke JF, Fuhrman JA, Lennon JT, Middelboe M, Stock CA, Suttle CA, Wilson WH, Wommack EK, Wilhelm SW, Weitz JS (2016) Re-examining the relationship between virus and microbial cell abundances in the global oceans. \textit{Nature Microbiology} 1: 15024. \href{https://lennonlab.github.io/assets/publications/Wiggington_etal_2016.pdf}{(pdf)}

\item Hall EK, Schoolmaster DR, Amado AM, Stets EG, Lennon JT, Domine L, Cotner JB (2016) Scaling relationships among drivers of aquatic respiration: from the smallest to the largest freshwater ecosystems. \textit{Inland Waters} 6: 1–10. \href{https://lennonlab.github.io/assets/publications/Hall_etal_2016.pdf}{(pdf)}

\item Kinsman-Costello LE, Hamilton SK, O'Brien J, Lennon JT (2016) Phosphorus release from the drying and reflooding of diverse wetland sediments. \textit{Biogeochemistry} 130: 159–176. \href{https://lennonlab.github.io/assets/publications/Kinsman-Costello_etal_2016.pdf}{(pdf)}

\item Martiny JBH, Jones SE, Lennon JT, Martiny AC (2015) Microbiomes in light of traits: a phylogenetic perspective. \textit{Science} 350: aac9323. \href{https://lennonlab.github.io/assets/publications/Martiny_etal_2015.pdf}{(pdf)}

\item Bier RL, Bernhardt ES, Boot CM, Graham EB, Hall EK, Lennon JT, Nemergut D, Osborne BB, Ruiz-Gonzalez C, Schimel JP, Waldrop MP, Wallenstein MD (2015) Linking microbial community structure and microbial processes: an empirical and conceptual overview. \textit{FEMS Microbiology Ecology} 91: fiv113. \href{https://lennonlab.github.io/assets/publications/Bier_etal_2015.pdf}{(pdf)}

\item Shoemaker WR, Muscarella ME, Lennon JT (2015) Genome sequence of the soil bacterium \textit{Janthinobacterium} sp. KBS0711. \textit{Genome Announcements} 3: e00689-15. \href{https://lennonlab.github.io/assets/publications/Shoemaker_etal_2015.pdf}{(pdf)}

\item Treseder KK, Lennon JT (2015) Fungal traits that drive ecosystem dynamics. \textit{Microbiology and Molecular Biology Reviews} 79: 243–262. \href{https://lennonlab.github.io/assets/publications/Treseder_Lennon_2015.pdf}{(pdf)}

\item Solomon CT, Jones SE, Weidel BC, Buffam I, Fork ML, Karlsson J, Larsen S, Lennon JT, Read JS, Sadro S, Saros JE (2015) Ecosystem consequences of changing inputs of terrestrial dissolved organic matter to lakes: current knowledge and future challenges. \textit{Ecosystems} 18: 376–389. \href{https://lennonlab.github.io/assets/publications/Solomon_etal_2015.pdf}{(pdf)}

\item Weitz JS, Stock CA, Wilhelm SW, Bourouiba L, Buchan A, Coleman ML, Follows MJ, Fuhrman JA, Jover LF, Lennon JT, Middelboe M, Sonderegger DL, Suttle CA, Taylor BP, Thingstad TF, Wilson WH, Wommack EK (2015) A multitrophic model to quantify the effects of marine viruses on microbial food webs and ecosystem processes. \textit{The ISME Journal} 9: 1352–1364. \href{https://lennonlab.github.io/assets/publications/Weitz_etal_2015.pdf}{(pdf)}

\item Aanderud ZT, Jones SE, Fierer N, Lennon JT (2015) Resuscitation of the rare biosphere contributes to pulses of ecosystem activity. \textit{Frontiers in Microbiology} 6: 24. \href{https://lennonlab.github.io/assets/publications/Aanderud_etal_2015.pdf}{(pdf)}

\item Jones SE, Lennon JT (2015) A test of the subsidy-stability hypothesis: effects of terrestrial carbon in aquatic ecosystems. \textit{Ecology} 96: 1550–1560. \href{https://lennonlab.github.io/assets/publications/Jones_Lennon_2015.pdf}{(pdf}, \href{https://esapubs.org/archive/ecol/E096/138/}{supplement}, \\ \href{https://lennonlab.github.io/assets/publications/Jones_Lennon_2016_ESABulletin.pdf}{ESA Bulletin Photo Gallery)}

\item Rocca JD, Hall EK, Lennon JT, Evans SE, Waldrop MP, Cotner JB, Nemergut DR, Graham EB, Wallenstein MD (2015) Relationships between protein-encoding gene abundance and corresponding process are commonly assumed yet rarely observed. \textit{The ISME Journal} 9: 1693–1699. \href{https://lennonlab.github.io/assets/publications/Rocca_etal_2015.pdf}{(pdf)}

\item Muscarella ME, Bird KC, Larsen ML, Placella SA, Lennon JT (2014) Phosphorus resource heterogeneity in microbial food webs. \textit{Aquatic Microbial Ecology} 73: 259–272. \href{https://lennonlab.github.io/assets/publications/Muscarella_etal_2014.pdf}{(pdf)}

\item Peralta AL, Stuart D, Kent AD, Lennon JT (2014) A social-ecological framework for \\ "micromanaging" microbial services. \textit{Frontiers in Ecology and the Environment} 12: 524–531. \href{https://lennonlab.github.io/assets/publications/Peralta_etal_2014.pdf}{(pdf}, \href{https://lennonlab.github.io/assets/publications/Peralta_etal_2014_Suppl.pdf}{supplement}, \href{https://lennonlab.github.io/assets/publications/Peralta_etal_2014_Cover.pdf}{cover)}

\item Krause S, Le Roux X, Niklaus PA, Van Bodegom P, Lennon JT, Bertilsson S, Grossart HP, Philippot L, Bodelier P (2014) Trait-based approaches for understanding microbial biodiversity and ecosystem functioning. \textit{Frontiers in Microbiology} 5: 251. \href{https://lennonlab.github.io/assets/publications/Krause_etal_2014.pdf}{(pdf)}

\item terHorst CP, Lennon JT, Lau JA (2014) The relative importance of rapid evolution for plant-soil feedbacks depend on ecological context. \textit{Proceedings of the Royal Society B} 281: 20140028. \href{https://lennonlab.github.io/assets/publications/terHorst_etal_2014.pdf}{(pdf}, \href{https://lennonlab.github.io/assets/publications/terHorst_etal_2014_correction.pdf}{correction)}

\item Dzialowski AR, Rzepecki M, Kostrzewska-Szlakowska I, Lennon JT, Kalinowska K, Palash A (2014) Are the abiotic and biotic characteristics of aquatic mesocosms representative of in situ conditions? \textit{Journal of Limnology} 73: 603–612. \href{https://lennonlab.github.io/assets/publications/Dzialowski_etal_2014.pdf}{(pdf)}

\item Bertilsson S, Burgin A, Carey CC, Fey SB, Grossart HP, Grubisic L, Jones I, Kirillin G, Lennon JT, Shade A, Smyth RL (2013) The under-ice microbiome of seasonally frozen lakes. \textit{Limnology and Oceanography} 58: 1998–2012. \href{https://lennonlab.github.io/assets/publications/Bertilsson_etal_2013.pdf}{(pdf)}

\item Lennon JT, Hamilton SK, Muscarella ME, Grandy AS, Wickings K, Jones SE (2013) A source of terrestrial organic carbon to investigate the browning of aquatic ecosystems. \textit{PLOS ONE} 8: e75771. \href{https://lennonlab.github.io/assets/publications/Lennon_etal_2013.pdf}{(pdf)}

\item Ponsero AJ, Chen F, Lennon JT, Wilhelm SW (2013) Complete genome sequence of a non-lysogenizing cyanobacterial siphoviridae. \textit{Genome Announcements} 1(4): e00472-13. \href{https://lennonlab.github.io/assets/publications/Ponsero_etal_2013.pdf}{(pdf)}

\item Lauber CL, Ramirez KS, Aanderud ZT, Lennon JT, Fierer N (2013) Temporal variability in soil microbial communities across land-use types. \textit{The ISME Journal} 7: 1641–1650. \href{https://lennonlab.github.io/assets/publications/Lauber_etal_2013.pdf}{(pdf)}

\item Aanderud ZT, Jones SE, Schoolmaster DR, Fierer N, Lennon JT (2013) Sensitivity of soil respiration and microbial communities to altered snowfall. \textit{Soil Biology and Biochemistry} 57: 217–227. \href{https://lennonlab.github.io/assets/publications/Aanderud_etal_2013.pdf}{(pdf)}

\item Shade A, Peter H, Allison S, Baho D, Berga M, Burgmann H, Huber D, Langenheder S, Lennon JT, Martiny JBH, Matulich K, Schmidt TM, Handelsman J (2012) Fundamentals of microbial community resistance and resilience. \textit{Frontiers in Microbiology} 3: 417. \href{https://lennonlab.github.io/assets/publications/Shade_etal_2012.pdf}{(pdf)}

\item Lau JA, Lennon JT (2012) Rapid responses of soil microorganisms improve plant fitness in novel environments. \textit{Proceedings of the National Academy of Sciences of the United States of America} 109: 14058–14062. \href{https://lennonlab.github.io/assets/publications/Lau_Lennon_2012.pdf}{(pdf}, \href{https://lennonlab.github.io/assets/publications/Lau_Lennon_2012_Suppl.pdf}{supplement}, \href{https://www.sciencedaily.com/releases/2012/08/120814110957.htm}{press release}, \href{http://www.bacteriofiles.com/2012/09/bacteriofiles-micro-edition-103.html}{pod cast}, \\ \href{https://f1000.com/prime/717978110}{F1000 recommendation}, \href{https://www.pnas.org/doi/10.1073/pnas.2118690118}{Correction)}

\item Lennon JT, Aanderud ZA, Lehmkuhl BK, Schoolmaster DR (2012) Mapping the niche space of soil microorganisms using taxonomy and traits. \textit{Ecology} 93: 1867–1879. \href{https://lennonlab.github.io/assets/publications/Lennon_etal_2012.pdf}{(pdf}, \\ \href{https://esapubs.org/archive/ecol/E093/165/}{Supplement}, \href{https://lennonlab.github.io/assets/publications/Lennon_etal_2012_ESABull.pdf}{ESA Bulletin Photo Gallery)}

\item Burgin AJ, Hamilton SK, Jones SE, Lennon JT (2012) Denitrification by sulfur-oxidizing bacteria in a eutrophic lake. \textit{Aquatic Microbial Ecology} 66: 283–293. \href{https://lennonlab.github.io/assets/publications/Burgin_etal_2012.pdf}{(pdf)}

\item O'Brien JM, Hamilton SK, Kinsman-Costello LE, Lennon JT, Ostrom NE (2012) Nitrogen transformations in a through-flow wetland revealed using whole-ecosystem pulsed 15N additions. \textit{Limnology and Oceanography} 57: 221–234. \href{https://lennonlab.github.io/assets/publications/OBrien_etal_2012.pdf}{(pdf)}

\item Treseder KK, Balser TC, Bradford MA, Brodie EL, Eviner VT, Hofmockel KS, Lennon JT, Levine UY, MacGregor BJ, Pett-Ridge J, Waldrop MP (2012) Integrating microbial ecology into ecosystem models. \textit{Biogeochemistry} 109: 7–18. \href{https://lennonlab.github.io/assets/publications/Treseder_etal_2012.pdf}{(pdf)}

\item Lau JA, Lennon JT (2011) Evolutionary ecology of plant-microbe interactions: soil microbial structure alters natural selection on plant traits. \textit{New Phytologist} 192: 215–224. \href{https://lennonlab.github.io/assets/publications/Lau_Lennon_2011.pdf}{(pdf)}

\item Aanderud ZT, Lennon JT (2011) Validation of heavy-water stable isotope probing for the characterization of rapidly responding soil bacteria. \textit{Applied and Environmental Microbiology} 77: 4589–4596. \href{https://lennonlab.github.io/assets/publications/Aanderud_Lennon_2011.pdf}{(pdf)}

\item Fierer N, Lennon JT (2011) The generation and maintenance of diversity in microbial communities. \textit{American Journal of Botany} 98: 439–448. \href{https://lennonlab.github.io/assets/publications/Fierer_Lennon_2011.pdf}{(pdf)}

\item Thum RA, Lennon JT (2010) Comparative ecological niche models predictive the invasive spread of variable-leaf milfoil (\textit{Myriophyllum heterophyllum}) and its potential impact on closely related native species. \textit{Biological Invasions} 12: 133–143. \href{https://lennonlab.github.io/assets/publications/Thum_Lennon_2010.pdf}{(pdf)}

\item Jones SE, Lennon JT (2009) Evidence for limited microbial transfer of methane in a planktonic food web. \textit{Aquatic Microbial Ecology} 58: 45–53. \href{https://lennonlab.github.io/assets/publications/Jones_Lennon_2009.pdf}{(pdf)}

\item Lennon JT, Martiny JBH (2008) Rapid evolution buffers ecosystem impacts of viruses in a microbial food web. \textit{Ecology Letters} 11: 1177–1188. \href{https://lennonlab.github.io/assets/publications/Lennon_Martiny_2008.pdf}{(pdf}, \href{https://lennonlab.github.io/assets/publications/Lennon_Martiny_2008_Suppl.pdf}{supplement)}

\item Lennon JT, Cottingham KL (2008) Microbial productivity in variable resource environments. \textit{Ecology} 84: 1001–1014. \href{https://lennonlab.github.io/assets/publications/Lennon_Cottingham_2008.pdf}{(pdf}, \href{https://lennonlab.github.io/assets/publications/Lennon_Cottingham_2008_Suppl.pdf}{Supplement}, \href{https://lennonlab.github.io/assets/publications/Lennon_Cottingham_2008_ESABull.pdf}{ESA Bulletin Photo Gallery)}

\item Lennon JT, Khatana SAM, Marston MF, Martiny JBH (2007) Is there a cost of virus resistance in marine cyanobacteria? \textit{The ISME Journal} 1: 300–312. \href{https://lennonlab.github.io/assets/publications/Lennon_etal_2007.pdf}{(pdf}, \href{https://lennonlab.github.io/assets/publications/Lennon_etal_2007_ISMECov.jpg}{featured article)}

\item Lennon JT (2007) Diversity and metabolism of marine bacteria cultivated on dissolved DNA. \textit{Applied and Environmental Microbiology} 73: 2799–2805. \href{https://lennonlab.github.io/assets/publications/Lennon_2007.pdf}{(pdf)}

\item Reyns NB, Langenheder S, Lennon J (2007) Specialization vs. diversification: a trade-off for young scientists? \textit{Eos} 88: 343. \href{https://lennonlab.github.io/assets/publications/Reyns_etal_2007.pdf}{(pdf)}

\item Dzialowski AD, Lennon JT, Smith VH (2007) Food web structure provides biotic resistance against plankton invasion attempts. \textit{Biological Invasions} 9: 257–256. \href{https://lennonlab.github.io/assets/publications/Dzialowski_etal_2007.pdf}{(pdf)}

\item Lennon JT, Faiia AM, Feng X, Cottingham KL (2006) Relative importance of CO\textsubscript{2} recycling and CH\textsubscript{4} pathways in lake food webs along a terrestrial carbon gradient. \textit{Limnology and Oceanography} 51: 1602–1613. \href{https://lennonlab.github.io/assets/publications/Lennon_etal_2006.pdf}{(pdf}, \href{https://lennonlab.github.io/assets/publications/Lennon_etal_2006_Suppl.pdf}{supplement)}

\item Thum RA, Lennon JT (2006) Is hybridization responsible for invasive growth of non-indigenous water-milfoils? \textit{Biological Invasions} 84: 1061–1066. \href{https://lennonlab.github.io/assets/publications/Thum_Lennon_2006.pdf}{(pdf)}

\item Cottingham KL, Lennon JT, Brown BL (2005) Regression versus ANOVA. \textit{Frontiers in Ecology and the Environment} 3: 358. \href{https://lennonlab.github.io/assets/publications/Cottingham_etal_2005b.pdf}{(pdf)}

\item Cottingham KL, Lennon JT, Brown BL (2005) Knowing when to draw the line: designing more informative ecological experiments. \textit{Frontiers in Ecology and the Environment} 3: 145–152. \href{https://lennonlab.github.io/assets/publications/Cottingham_etal_2005a.pdf}{(pdf}, \href{https://lennonlab.github.io/assets/publications/Cottingham_etal_2005a_Suppl.pdf}{supplement)}

\item Lennon JT, Pfaff LE (2005) Source and supply of terrestrial carbon affects aquatic microbial metabolism. \textit{Aquatic Microbial Ecology} 39: 107–119. \href{https://lennonlab.github.io/assets/publications/Lennon_Pfaff_2005.pdf}{(pdf)}

\item Thum RA, Lennon JT, Connor J, Smagula AP (2005) A DNA fingerprinting approach for distinguishing native and non-native milfoils. \textit{Lake and Reservoir Management} 21: 1–6. \href{https://lennonlab.github.io/assets/publications/Thum_etal_2005.pdf}{(pdf)}

\item Lennon JT (2004) Experimental evidence that terrestrial carbon subsidies increase CO\textsubscript{2} flux from lake ecosystems. \textit{Oecologia} 138: 584–591. \href{https://lennonlab.github.io/assets/publications/Lennon_2004.pdf}{(pdf)}

\item Lennon JT, Smith VH, Dzialowski AR (2003) Invasibility of plankton food webs along a trophic state gradient. \textit{Oikos} 102: 191–203. \href{https://lennonlab.github.io/assets/publications/Lennon_etal_2003.pdf}{(pdf)}

\item Dzialowski AR, Lennon JT, O'Brien WJ, Smith VH (2003) Predator-induced phenotypic plasticity in the exotic cladoceran \textit{Daphnia lumholtzi}. \textit{Freshwater Biology} 48: 1593–1602. \href{https://lennonlab.github.io/assets/publications/Dzialowski_etal_2003.pdf}{(pdf}, \href{https://lennonlab.github.io/assets/publications/Dzialowski_etal_2003_Cover.pdf}{cover)}

\item Cottingham KL, Brown BL, Lennon JT (2001) Biodiversity may regulate the temporal variability of ecological systems. \textit{Ecology Letters} 4: 72–85. \href{https://lennonlab.github.io/assets/publications/Cottingham_etal_2001.pdf}{(pdf)}

\item Lennon JT, Smith VH, Williams K (2001) Influence of temperature on exotic \textit{Daphnia lumholtzi} and implications for invasion success. \textit{Journal of Plankton Research} 23: 425–434. \href{https://lennonlab.github.io/assets/publications/Lennon_etal_2001.pdf}{(pdf)}

\item deNoyelles FJ, Wang SH, Meyer JO, Huggins DG, Lennon JT, Kolln WS, Randtke SJ (1999) Water quality issues in reservoirs: some considerations from a study of a large reservoir in Kansas. \textit{Proceedings of the 49th Annual Environmental Engineering Conference, University of Kansas, Lawrence}. \href{https://lennonlab.github.io/assets/publications/deNoyelles_etal_1999.pdf}{(pdf)}

\end{etaremune}

\section*{Invited Keynote, Symposium, and Conference Presentations}
\vspace{-1.25em} % Adjust this value as needed 
\noindent
\begin{longtable}{@{}p{3em}@{\hspace{1.5em}}p{0.87\textwidth}@{}}

2026 & Invited workshop speaker, “Coarse-graining microbial ecology: from genes to physiological strategies to communities across environments”, Kavli Institute of Theoretical Physics (KITP), Santa Barbara, CA, USA \\
2025 & Invited symposium speaker, “Climate change and health: from micro to macro”, MAC-EPID, University of Michigan, Ann Arbor, MI, USA \\
2025 & Keynote speaker, Applied and Environmental Sciences (AES) Retreat, American Society of Microbiology \\
2025 & Invited speaker, “Synthetic biology enabling tools”, ASBMB, Chicago, IL, USA \\
2025 & Invited speaker, “Evolutionary ecology of dormancy in a community context”, Ecological Society of America, Baltimore, MD, USA \\
2025 & Invited speaker, Council on Microbial Sciences (COMS) meeting, American Society of Microbiology \\
2024 & Invited speaker, Union Session: “One Health, Microbes, and Geosciences”, AGU, Washington, DC, USA \\
2024 & Plenary speaker, Chinese Association of Microbial Ecology, Qingdao, China \\
2024 & Plenary speaker, International Symposium on Soil Microbiomes and Soil Health, Yangling, China \\
2024 & Plenary speaker, mLife Research Conference, Shenzhen, China \\
2024 & Invited speaker, “Harnessing transformational technologies symposium”, Los Alamos National Laboratory and National Academies of Sciences, Santa Fe, NM, USA \\
2024 & Invited speaker, mSystems Thinking Series, “Critical concepts in microbial dormancy”, Virtual \\
2024 & Invited speaker, ASM Public and Scientific Affairs Committee (PSAC) Meeting, Virtual \\
2023 & Plenary speaker, “Critical phenomena and challenges emerging from dormancy”, University of Frankfurt, Germany \\
2023 & Invited speaker, AGU, “Linkage between geosciences and biothreats”, San Francisco, CA, USA \\
2023 & Invited workshop speaker, “Quantitative principles in microbial physiology”, ICTP, Trieste, Italy \\
2023 & Invited speaker, “Microbes and climate change”, ASM, Houston, TX, USA \\
2023 & Invited panelist, “Communicating about climate change with a lay audience”, ASM, Houston, TX, USA \\
2022 & Invited speaker, NAS Board on Life Sciences, “Harnessing microbial diversity for the bioeconomy”, Virtual \\
2022 & Invited moderator, U.S. Congressional briefing, “Big Problems, Tiny Solutions”, Virtual \\
2022 & Roger D. Milkman Endowed Lecturer, Marine Biological Laboratory, Woods Hole, MA, USA \\
2022 & Invited speaker, “Communicating about climate change”, ASM, Washington, DC, USA \\
2022 & Plenary speaker, MEEC, University of Kansas, Lawrence, KS, USA \\
2022 & Invited speaker, HIM Workshop on Dormancy, Universität Bonn, Germany \\
2022 & Distinguished/Waksman Foundation Lecture, ASM Student Chapters, Virtual \\
2022 & Invited workshop speaker, MCC Workshop, University of Chicago, IL, USA \\
2022 & Invited workshop speaker, “Quantitative biology of non-growing microbes”, KITP, Santa Barbara, CA, USA \\
2022 & Opening speaker, 4th Workshop on Microbial Life under Energy Limitation, Sandbjerg Castle, Denmark \\
2022 & Distinguished lecturer, Indiana Branch ASM Annual Meeting, West Lafayette, IN, USA \\
2021 & Invited workshop speaker, “Ecology and Evolution of Microbial Communities”, KITP, Santa Barbara, CA, USA \\
2021 & Invited speaker, ESA Special Session: “Microbial Connectivity across Ecotones”, Long Beach, CA, USA \\
2020 & Keynote speaker, “Microbial Ecology \& Evolution (MEE)”, Virtual \\
2019 & Invited workshop speaker, “Deciphering the Microbiome”, NSF, Alexandria, VA, USA \\
2019 & Plenary speaker, “Evolutionary consequences of dormancy”, TU Berlin, Germany \\
2019 & Invited speaker, ESA Inspire Session, “Microbial ecology at scale”, Louisville, KY, USA \\
2019 & Invited Lecturer, Microbial Biodiversity Course, Marine Biological Laboratory, MA, USA \\
2018 & Invited Lecturer, STAMPS and Microbial Biodiversity Courses, Marine Biological Laboratory, MA, USA \\
2018 & Plenary speaker, MEEC, Michigan State University, Hickory Corners, MI, USA \\
2018 & Invited speaker, ESA Special Session: “Plant microbiomes in a changing world”, New Orleans, LA, USA \\
2017 & Invited speaker, Argonne Soil Metagenomics Meeting, Argonne National Lab, IL, USA \\
2017 & Invited speaker, GRC Microbial Population Biology and GRC Applied \& Environmental Microbiology, NH \& MA, USA \\
2017 & Invited speaker, ESA and ASM Special Sessions on eco-evo feedbacks and trait ecology \\
2016 & Invited workshop speaker, NASEM Microbiomes of the Built Environment, Irvine, CA, USA \\
2016 & Invited speaker, AGU Special Session: Biogeochemical cycles, San Francisco, CA, USA \\
2016 & Plenary speaker, Translational Plant Science Program, Virginia Tech, Blacksburg, VA, USA \\
2016 & Invited workshop, “Skin microbiome – untold stories”, Society of Cosmetic Chemists, Orlando, FL, USA \\
2015 & Invited speaker, EDAMAME Course, Michigan State University, Hickory Corners, MI, USA \\
2015 & Introductory speaker, Summer Soil Institute, Colorado State University, Fort Collins, CO, USA \\
2015 & Invited speaker, ESA Special Session: “Rewetting dry soil”, Baltimore, MD, USA \\
2014 & Invited speaker and panelist, ISME, Seoul, Korea \\
2014 & Invited speaker, ESA Special Symposium: Microbial community ecology, Sacramento, CA, USA \\
2014 & Invited speaker, EDAMAME Course, Michigan State University \\
2013 & Keynote speaker, Argonne and EuroEEFG workshops, IL, USA \& Wageningen, Netherlands \\
2013 & Invited speaker, ESA, ASLO, SGM, and Canadian Society for Ecology and Evolution \\
2012 & Invited speaker, Aarhus University, Denmark \\
2012 & Intro speaker, ESA Special Session: Global browning of inland waters, Portland, OR, USA \\
2010 & Invited speaker, GRC Marine Microbes and Argonne Meetings, NH \& IL, USA \\
2009 & Invited speaker, SCOR and Plant Virus Ecology Meetings, USA \& Italy \\
2008 & Invited speaker, ESA Special Session: Ecological theory, Milwaukee, WI, USA \\
2006 & Invited speaker, ISME Special Session: Microbial Communities, Vienna, Austria \\
2007 & Invited speaker, Arctic ecosystems workshop, Abisko, Sweden \\
\end{longtable}


\section*{Invited Seminars}
\vspace{-1.25em} % Adjust this value as needed 
\noindent
\begin{longtable}{@{}p{3em}@{\hspace{1.5em}}p{0.87\textwidth}@{}}
2025 & University of Illinois, Department of Microbiology \\
2025 & Yale University, Department of Ecology and Evolutionary Biology \\
2025 & University of Connecticut, Department of Ecology and Evolutionary Biology \\
2025 & University of Alaska Fairbanks, Department of Biology and Wildlife \\
2025 & University of Maryland, Department of Biology \\
2024 & Ocean University of China, Institute of Evolution and Marine Biodiversity \\
2024 & Northwest A\&F University, Department of Environmental Science and Engineering \\
2024 & University of Southern California, Department of Biological Sciences \\
2024 & University of California San Diego, Department of Ecology, Behavior, and Evolution \\
2024 & Carnegie Institute of Science \\
2024 & Pennsylvania State University, One Health Microbiome Center Seminar Series \\
2024 & Lehigh University, Earth \& Environmental Sciences \\
2023 & ETH Zürich, Institute of Microbiology \\
2023 & University of Vienna, Department of Microbiology and Ecosystem Science \\
2023 & Eawag, Swiss Federal Institute of Aquatic Science and Technology \\
2023 & ETH Zürich, Centre for Origin and Prevalence of Life (COPL) \\
2023 & University of Aberdeen, School of Biological Sciences \\
2023 & Frankfurt Institute for Advanced Studies, CMMS \\
2023 & National Science Foundation, Climate Change Coordinating Committee (C4) \\
2023 & R.D. Holt Seminar Series, University of Florida \\
2022 & InterActive Biomes Group, CSIRO, Australia \\
2022 & University of Hawaii, Pacific Biosciences Research Center \\
2022 & University of Florida, Department of Biology \\
2020 & Dartmouth College; Ecology, Evolution, Ecosystems and Society (EEES) \\
2019 & University of North Carolina Chapel Hill, Department of Microbiology and Immunology \\
2018 & University of Wisconsin, Distinguished Lecture in Microbiology \\
2018 & Pennsylvania State University, Plant Pathology and Environmental Microbiology \\
2018 & Purdue University, Microbiome Seminar Series \\
2017 & Yale University, Department of Ecology and Evolutionary Biology \\
2017 & University of Georgia, Odum School of Ecology \\
2017 & German Centre for Integrative Biodiversity Research (iDiv) \\
2017 & University of Tennessee, Department of Microbiology \\
2017 & University of Idaho, Department of Biological Sciences \\
2016 & MIT, Civil and Environmental Engineering \\
2016 & University of Minnesota, Ecology, Evolution, and Behavior \\
2016 & University of British Columbia, BioDiversity Research Centre Seminar \\
2016 & Michigan State University, Plant, Soil, and Microbial Sciences \\
2016 & Uppsala University, Ecology and Genetics \\
2016 & Hope College, Biology Department \\
2016 & University of Montana, Flathead Lake Biological Station \\
2016 & University of Montana, Cell, Molecular and Microbial Biology \\
2016 & Montana State University, Microbiology and Immunology \\
2015 & Duke University, University Program in Ecology \\
2015 & East Carolina University, Biology Department \\
2015 & Hobart and William Smith Colleges \\
2015 & Indiana University, CISAB \\
2015 & Indiana University, Advance College Project \\
2015 & Vietnam National University, Department of Microbiology \\
2014 & University of Tennessee, EEB \\
2014 & University of Louisville, Biology Department \\
2014 & University of Illinois, Ecology, Evolution, and Conservation Biology \\
2014 & Indiana University East and Earlham College \\
2014 & University of Kentucky, Plant \& Soil Sciences \\
2014 & Loyola University Chicago, Biology Department \\
2014 & Miami University, Ecology, Evolution, and Environmental Biology \\
2014 & Purdue University, Biological Sciences, EEB \\
2013 & University of Texas at Austin, Integrative Biology \\
2013 & University of Oregon, Ecology and Evolutionary Biology \\
2013 & UC Santa Barbara, Ecology, Evolution, and Marine Biology \\
2013 & The Netherlands Institute of Ecology (NIOO-KNAW) \\
2013 & University of Michigan, EEB \\
2012 & Virginia Tech, Biological Sciences \\
2012 & Northwestern University, Biological Sciences \\
2012 & University of Jyväskylä, Finland \\
2012 & University of Quebec at Montreal, Canada \\
2011 & Indiana University, Biology Department \\
2011 & University of Massachusetts, Amherst, Microbiology \\
2011 & Oregon State University, CGRB \\
2011 & California Academy of Sciences \\
2010 & Michigan Technological University, Forest Resources and Environmental Science \\
2010 & Michigan State University, Ecosystems Biogeochemistry \\
2010 & Wright State University, Biology and Earth \& Environmental Sciences \\
2009 & University of Illinois, Ecology, Evolution, and Conservation Biology \\
2008 & University of Illinois at Springfield, Merck Seminar \\
2008 & Western Michigan University, Biological Sciences \\
2007 & University of Tennessee, Knoxville, Haines-Morris Series \\
2007 & Grand Valley State University, Annis Water Resources Institute \\
2005 & Michigan State University, Microbiology and Molecular Genetics \\
2005 & Kellogg Biological Station, Michigan State University \\
2005 & Dartmouth College, Earth Sciences \\
2004 & UC Berkeley, Division of Ecosystem Sciences \\
2004 & Brown University, Ecology \& Evolutionary Biology \\
2002 & Colby-Sawyer College, Biology Program \\
\end{longtable}


\section*{Organizer for Symposia, Workshops, and Conferences}
\vspace{-1.25em} % Adjust this value as needed 
\noindent
\begin{longtable}{@{}p{3em}@{\hspace{1.5em}}p{0.87\textwidth}@{}}

2025 & Co-organizer, “Cancer dormancy and therapy resistance: from models to the clinic” Rome, Italy \\
2025 & Co-chair, colloquium steering committee “Microbes, human health, and climate change” American Academy of Microbiology, Washington, DC, USA \\
2024 & Organizer, “Enhancing methane mitigation strategies via methanogenesis and methanotrophy” Microbe meeting, ASM, Atlanta, Georgia, USA \\
2023 & Co-organizer, “Critical phenomena and challenges emerging from dormancy” Goethe University, Frankfurt, Germany \\
2023 & Co-organizer, “Dormancy, rarity, and community dynamics” Ecological Society of America, Portland, Oregon, USA \\
2022 & Co-chair, colloquium steering committee “Microbes in models: steps for integrating microbial activity into climate models” AAM, Washington, DC, USA \\
2022 & Mini-conference co-organizer, “Climate change and microbes” Microbe meeting, ASM, Washington, DC, USA \\
2020 & Symposium organizer, “Birth and death: a quantitative approach to microbial populations” ASM, Chicago, Illinois, USA \\
2019 & Plenary organizer, “Evolution in the wild” Microbe meeting, ASM, San Francisco, California, USA \\
2018 & Symposium co-organizer, “Assembly and function of microbial communities: a trait-based approach” Microbe meeting, ASM, Atlanta, Georgia, USA \\
2017 & Retreat co-organizer, Microbial Ecology and Evolution Track, ASM, Washington, DC, USA \\
2017 & Special symposium co-organizer, “Eco-evo feedbacks in microbial communities” ASM, New Orleans, Louisiana, USA \\
2015 & Special session co-organizer, “Trait-based ecology at the microscale” Ecological Society of America, Baltimore, Maryland, USA \\
2014 & Special session co-organizer, “Microbially mediated ecosystem services: The good, the bad and the ugly” Joint Aquatic Sciences Meeting, Portland, Oregon, USA \\
2013 & Organizing committee, First Israel-U.S. Kavli Frontiers of Science symposium, Israel Academy of Sciences, U.S. National Academy of Sciences, Irvine, California \\
2013 & Special symposium organizer, “Next generation of ecological indicators: defining which microbial properties matter most to ecosystem function and how to measure them” ESA, Minneapolis, Minnesota, USA \\
2012 & Co-investigator, John Wesley Powell Center, “Next generation of ecological indicators...” Fort Collins, Colorado, USA (2012--2015) \\
2012 & Round table co-organizer, “Frontiers in microbial ecosystem science: energizing the research agenda” ISME, Copenhagen, Denmark \\
2012 & Invited co-convener, “The unknowns: rare ones and unculturables” ISME, Copenhagen, Denmark \\
2012 & Workshop co-organizer, “Answering ecological questions with metagenomic sequencing” ESA, Portland, Oregon, USA \\
2011 & Special symposium organizer, “Micro-managing the planet: the role of microbial ecology in earth stewardship” ESA, Austin, Texas, USA \\
2010 & Special session co-organizer, “Micro-managing the planet: the role of microbial ecology in earth stewardship” ESA, Pittsburgh, Pennsylvania, USA \\
2002 & Special session co-organizer, “Ecological implications of terrestrial inputs into lakes and ponds” ASLO, Victoria, British Columbia, Canada \\
\end{longtable}

\section*{Invited Participant: Workshops, Roundtables, and Synthesis Groups}
\vspace{-1.25em} % Adjust this value as needed 
\noindent
\begin{longtable}{@{}p{2cm}@{\hspace{1em}}p{14cm}@{}}
2024 & Invited colloquium participant, “Impacts of the changing climate on water, water-borne pathogens, and human health colloquium” American Academy of Microbiology and the American Geophysical Union, Washington, DC, USA \\
2024 & Invited participant, “Dormancy in soil microbiomes” Northwest University, Xi’An, China \\
2024 & Invited moderator, American Society of Microbiology, Meet the Policymaker Series: National Climate Assessment, virtual \\
2024 & Invited workshop participant, “Developing a rapid, cost-effective, and information-rich metric of biodiversity resilience” German Centre for Integrative Biodiversity Research (iDiv), Leipzig, Germany \\
2024 & Invited participant, “Soil microbial strategies for climate mitigation” Oath Soil Life, Las Vegas, NV, USA \\
2023 & Invited workshop participant, “LIFE: Leveraging Innovation From Evolution”, virtual \\
2023 & Invited participant and opening remarks, “Microbiology and climate change communications workshop: opportunities and challenges” American Society for Microbiology, virtual \\
2023 & Invited panelist, “How to effectively communicate about climate change with a lay audience” American Society for Microbiology, Houston, TX, USA \\
2023 & Invited colloquium participant, “The role of microbes in mediating methane emissions – act today to prepare for tomorrow” American Academy of Microbiology and the American Geophysical Union, Washington, DC, USA \\
2022 & Invited workshop participant, “Integrating macro-ecology and macro-evolution for biodiversity assessment” Schoodic Institute, ME, USA \\
2022 & Invited workshop participant, “Understanding the rules of life: harnessing microbiomes for societal benefit”, Washington, DC, USA \\
2022 & Participant, Microbiome Centers Consortium, Chicago, IL, USA (virtual) \\
2017 & Invited workshop participant, “Patterns of microbial and macrobial diversity” German Centre for Integrative Biodiversity Research (iDiv), Leipzig, Germany \\
2017 & Invited workshop participant, “Continuum of persistence: ecology and function of persistent virus infections” Cascais, Portugal \\
2016 & Invited workshop participant, “PRO-MICROBES: Vision Theme meeting on the Microbiome” Marine Biological Laboratory, Woods Hole, MA, USA \\
2016 & Invited working group participant, NSF RCN, “Utilizing ongoing experiments to understand ecosystem sensitivity to precipitation change and drought” Sevilleta NWR, NM, USA \\
2015 & Invited workshop participant, “Biocomplexity” DARPA, Arlington, VA, USA \\
2014 & Invited workshop participant, “Advanced analysis of genomic data in microbial ecology research” NEON, Boulder, CO, USA \\
2014 & Invited workshop participant, “Ecological implications of synthetic biology” NSF/MIT/Woodrow Wilson Center, Emeryville, CA, USA \\
2012 & Invited workshop participant, National Academy of Sciences, German-American Kavli Frontiers of Science, Potsdam, Germany \\
2012 & Invited roundtable participant, “Frontiers in ecosystem science: energizing the research agenda” ESA, Portland, OR, USA \\
2011 & Invited working group participant, “Modeling viral effects on global carbon and biogeochemical cycles” NIMBioS, Knoxville, TN, USA (2011--2014) \\
2011 & Invited technical expert, “Comprehensive Environmental Assessment of ecological impacts of synthetic biology” Woodrow Wilson Center, Washington, DC, USA \\
2010 & Invited roundtable participant, “Resilience in microbial communities: toward prediction and comparison” ISME, Seattle, WA, USA \\
2010 & Invited workshop participant, “A synthesis of allochthonous vs autochthonous support in food webs” ASLO, Santa Fe, NM, USA \\
2009 & Invited workshop participant, “Role of viruses in marine ecosystems” SCOR, University of Delaware, Newark, DE, USA \\
2009 & Invited workshop participant, Plant Virus Ecology Network (PVEN), Ca’ Tron di Roncade, Italy \\
2008 & Invited workshop participant, SoilCritZone Early Stage Researcher training, Chania, Crete, Greece \\
2008 & Invited workshop participant, DOE JGI Microbial Genomics \& Metagenomics, Walnut Creek, CA, USA \\
2007 & Invited workshop participant, “LTER Genomics: Cross-site comparisons of microbial diversity and function” East Lansing, MI, USA \\
2007 & Invited workshop participant, DOE JGI Undergraduate Program in Genome Annotation, Walnut Creek, CA, USA \\
2007 & Invited workshop participant, Microscale Approaches to Macroscale Issues in Ecology, Washington, DC, USA \\
2007 & Invited workshop participant, Early Career Faculty in Ecoinformatics, SEEK, Albuquerque, NM, USA \\
2005 & Invited workshop participant, DIALOG VII, ASLO, Dauphin Island Sea Lab, AL, USA \\
\end{longtable}

\section*{Contributed Presentations}
\vspace{-0.5em}

{\setlength{\parskip}{0.3em}  % Tightens vertical spacing between items

\hangindent=2em Lennon JT (2025) Resuscitation-promoting factor (Rpf) terminates dormancy among diverse soil bacteria. American Society for Microbiology, Los Angeles, California, USA. \par

\hangindent=2em Nevermann Henrik D, Gros C, Lennon JT (2025) A game of life with dormancy. Deutsche Physikalische Gesellschaft, DPG (German Physical Society), Regensburg, Germany. \par

\hangindent=2em Nevermann Henrik D, Gros C, Lennon JT (2025) A game of life with dormancy. Dynamical Systems Applied on Biology and Natural Sciences (DSABNS), Naples, Italy. \par

\hangindent=2em Nevermann Henrik D, Gros C, Lennon JT (2025) A game of life with dormancy. Dynamic Days, Bremen, Germany. \par

\hangindent=2em Karakoç C, Lennon JT (2023) Evolution of survival through the lens of bioenergetics. AbGradCon2023, La Jolla, California, USA. \par

\hangindent=2em Webster KD, Schimmelmann A, Lennon JT (2023) Genomic insights into methane consumption in caves. Society for Industrial Microbiology and Biotechnology, Minneapolis, Minnesota, USA. \par

\hangindent=2em Karakoç C, Lennon JT (2023) Evolution of complex traits through the lens of bioenergetics. Ecological Society of America, Portland, Oregon, USA. \par

\hangindent=2em McMullen JG, Lennon JT (2023) Microbial mark-recapture: a novel approach to disentangle complex microbial lifestyles. Ecological Society of America, Portland, Oregon, USA. \par


\hangindent=2em Mueller E, Lennon JT (2023) Residence time as a control on the diversity and function of lake microbial communities. Ecological Society of America, Portland, Oregon, USA. \par

\hangindent=2em Moger-Reischer RZ, Glass JI, Wise KS, Sun L, Bittencourt DMC, Lynch M, Lennon JT (2021) Evolution of a minimal cell. Minimal Cell Workshop, La Jolla, California, USA (virtual). \par

\hangindent=2em Wang J, Hu A, Choi M, Tanentzap AJ, Liu J, Jang KS, Lennon JT, Liu Y, Soininen J, Lu X, Zhang Y, Shen J (2021) Quantifying microbial associations of dissolved organic matter under global change. American Geophysical Union, New Orleans, Louisiana, USA. \par

\hangindent=2em Fishman F, Lennon JT (2020) Macroevolutionary constraints on global microbial biodiversity. American Society for Microbiology, Chicago, Illinois, USA. \par

\hangindent=2em Schwartz DA, Lennon JT (2020) Viral manipulation of bacterial dormancy. American Society for Microbiology, Chicago, Illinois, USA. \par

\hangindent=2em Mueller, Lennon JT (2020) Physical complexity controls microbial abundance and function of microbiomes in 3D-printed gut bioreactors. American Society for Microbiology, Chicago, Illinois, USA. \par

\hangindent=2em Mueller EA, Lennon JT (2019) Physical complexity as a control on diversity and function of gut microbiomes. Ecological Society of America, Louisville, Kentucky, USA. \par

\hangindent=2em Behringer MG, Lennon JT (2019) Mutation accumulation during dormancy. American Society for Microbiology, San Francisco, California, USA. \par

\hangindent=2em Mueller EA, Lennon JT (2019) Physical complexity as a control on the abundance and metabolic activity of gut microbiomes. Purdue Microbiome Symposium, West Lafayette, Indiana, USA. \par

\hangindent=2em Wisnsoski NI, Leibold MA, Lennon JT (2019) Dormancy in metacommunities: when can temporal dispersal maintain diversity in variable landscapes? Society for Freshwater Science, Salt Lake City, Utah, USA. \par

\hangindent=2em Shoemaker WR, Lennon JT (2018) Dormancy constrains the rate and direction of adaptive evolution. Population, Evolutionary, and Quantitative Genetics Conference, Madison, Wisconsin, USA. \par

\hangindent=2em Shoemaker WR, Locey KJ, Lennon JT (2018) Reproducing global biodiversity estimates through evolutionary and biophysical theory. Theory in Biology Meeting, Boston, Massachusetts, USA. \par

\hangindent=2em Wisnoski NI, Lennon JT (2018) Dispersal and dormancy across ecosystems boundaries: bacterial diversity and function along a reservoir transect. Association for the Sciences of Limnology and Oceanography, Victoria, British Columbia, Canada. \par

\hangindent=2em Wisnoski NI, Lennon JT (2018) Contribution of “seed banks” to bacterioplankton community dynamics. Society for Freshwater Science, Detroit, Michigan, USA. \par

\hangindent=2em Shoemaker WR, Lennon JT (2017) The contribution of dormancy to microbial evolution. Society for Molecular Biology and Evolution, Austin, Texas, USA. \par

\hangindent=2em Benavidez MK, Milton K, Lennon JT, Wasserman M (2017) Interactions between gut microbial diversity and the endocrine system in wild howler monkeys (\textit{Alouatta palliata}). Midwest Primate Interest Group, Evanston, Illinois, USA. \par

\hangindent=2em Lennon JT, Jones SE (2017) Energy limitation in bacteria: a trait-based approach. Ecological Society of America, Portland, Oregon, USA. \par

\hangindent=2em Wisnoski NI, Lennon JT (2017) Dendritic metacommunities: a test of assembly using stream microbial communities. Ecological Society of America, Portland, Oregon, USA. \par

\hangindent=2em Salazar A, Lennon JT, Dukes JS (2017) Microbial activity is a better predictor of soil respiration than microbial biomass or composition. Ecological Society of America, Portland, Oregon, USA. \par

\hangindent=2em Fisk MC, Goswami S, Shan S, Lennon JT, See C, Yanai RD, Fahey TJ (2017) Processes mediating interactions of N and P availability in northern hardwood forests. Ecological Society of America, Portland, Oregon, USA. \par

\hangindent=2em Lilleskov E, Kane E, Chimner R, Koka R, Lennon JT, Lamit J, Ontl T, Romanowicz K, Wiedermann L, Veverica T, Daniels A (2017) Hydrology and plant functional groups alter carbon cycling in Sphagnum peatlands: the PEATcosm experiment. Society of Wetland Scientists, San Juan, Puerto Rico. \par

\hangindent=2em Lennon JT, Aanderud ZT (2016) A trait-based approach to understanding the microbial moisture niche. Third International Workshop on Biological Soil Crusts (BioCrust 3), Moab, Utah, USA. \par

\hangindent=2em Webster KD, Schimmelmann A, Lennon JT (2016) Diversity and function of methanotrophic bacteria in caves. American Geophysical Union, San Francisco, California, USA. \par

\hangindent=2em Thomas P, Kuo V, Bray SR, Lehmkuhl BK, Lennon JT. The effects of a resuscitation promoting factor (Rpf) on bacterial activity and plant biomass. Kentucky Academy of Science, Louisville, Kentucky, USA. \par

\hangindent=2em Beatty J, Wisnoski NI, Bray SR, Lennon JT. Residence time as driver of abundance, activity, and resource-use in complex microbial communities. Kentucky Academy of Science, Louisville, Kentucky, USA. \par

\hangindent=2em Lilleskov EA, Lamit JL, Lennon JT, Romanowicz KR, Tringe S, Kane ES, Potvin LR, Wiedermann L, Chimner R, Kolka R (2016) Fungal community response to water table and plant functional group manipulations in the PEATcosm experiment: evidence for the Gadgil effect? Mycological Society of America, Berkeley, California, USA. \par

\hangindent=2em Wisnoski NI, Lennon JT (2016) Community assembly processes differ between surface water and sediment-associated communities in stream networks. Ecological Society of America, Fort Lauderdale, Florida, USA. \par

\hangindent=2em Schimmelmann A, Lennon JT, Nguyen-Thuy D, Ta Hoa P, Drobniak A, Webster KD, Schimmelmann M (2016) Vietnam’s tropical karst is a sink for atmospheric methane greenhouse gas. 5th International Conference on Earth Science \& Climate Change, Bangkok, Thailand. \par

\hangindent=2em Locey KE, Lennon JT (2016) Scaling laws predict global microbial diversity. International Society of Microbial Ecology, Montreal, Canada. \par

\hangindent=2em Lennon JT, Cummins S, Miller KI, Schoolmaster DK (2016) Metabolic activity of the skin microbiome: is our first line of defense sleeping on the job? International Society of Microbial Ecology, Montreal, Canada. \par

\hangindent=2em Wisnoski NI, Lennon JT (2016) Local and regional processes in stream microbial community assembly. International Society of Microbial Ecology, Montreal, Canada. \par

\hangindent=2em Shoemaker WR, Lennon JT (2016) Microbial population-genomics under extreme starvation. International Society of Microbial Ecology, Montreal, Canada. \par

\hangindent=2em Lennon JT, SE Jones (2015) Ecological and evolutionary insight into the persistence of soil bacteria. Argonne Soil Metagenomics Workshop, Argonne National Laboratory, Lisle, Illinois, USA. \par

\hangindent=2em Lamit LJ, Lennon JT, Lilleskov EA (2015) Peatland microbial community responses to plant functional group, water table and depth. Argonne Soil Metagenomics Workshop, Lisle, Illinois, USA. \par

\hangindent=2em Wisnoski NI, Ward AS, Lennon JT (2015) Bacterial metacommunity structure across a stream network. Long-Term Ecological Research (LTER) All Scientist Meeting, Estes Park, Colorado, USA. \par

\hangindent=2em Lilleskov E, Kane E, Chmner R, Koka R, Lennon JT, Potvin L, Ontl T, Romanowicz K, Lamit JL, Daniels A (2015) PEATcosm: experimental insights into climate change effects on peatland carbon cycling and trace gas flux. Soil Science Society of America, Minneapolis, Minnesota, USA. \par

\hangindent=2em Peralta AP, Sun Y, Lennon JT (2015) Effects of crop diversity on plant-soil-microbial interactions. AFRI NIFA Fellows Program, Washington, District of Columbia, USA. \par

\hangindent=2em Peralta AP, Sun Y, Lennon JT (2015) Effects of crop diversity on plant-soil-microbial interactions. Long-Term Ecological Research (LTER) All Scientist Meeting, Estes Park, Colorado, USA. \par

\hangindent=2em Lau JA, Lennon JT, terHorst CP (2015) The interplay of ecology and evolution in aboveground\\-belowground response to environmental change. Ecological Society of America, Baltimore, Maryland, USA. \par

\hangindent=2em Aanderud ZT, Jones SE, Fierer N, Lennon JT (2015) Resuscitation of the rare biosphere contributes to pulses of ecosystem activity following soil rewetting. Ecological Society of America, Baltimore, Maryland, USA. \par

\hangindent=2em terHorst CP, Lennon JT, Lau JA (2015) Plant evolution in response to drought alters the structure and function of soil microbial communities. Ecological Society of America, Baltimore, Maryland, USA. \par

\hangindent=2em Martiny JBH, Jones SE, Lennon JT, Martiny AC (2015) Microbiomes in light of traits: a phylogenetic perspective. Ecological Society of America, Baltimore, Maryland, USA. \par

\hangindent=2em Lennon JT, Jones SE (2015) A trait-based approach to microbial dormancy. Ecological Society of America, Baltimore, Maryland, USA. \par

\hangindent=2em Locey KJ, Lennon JT (2015) Residence time: An overlooked constraint on community assembly and structure. Ecological Society of America, Baltimore, Maryland, USA. \par

\hangindent=2em Treseder KK, Lennon JT (2015) Fungal traits that drive ecosystem dynamics. Ecological Society of America, Baltimore, Maryland, USA. \par

\hangindent=2em Muscarella ME, Lennon JT (2015) Bacterial growth efficiency: do consumer and resource diversity influence the fate of carbon in aquatic ecosystems? Ecological Society of America, Baltimore, Maryland, USA. \par

\hangindent=2em Hall EK, Schoolmaster DK, Amado AM, Stets EG, Lennon JT, Domine L, Cotner JB (2015) Controls on aquatic respiration from the smallest to the largest freshwater ecosystems. Association for the Sciences of Limnology and Oceanography, Granada, Spain. \par

\hangindent=2em Larsen ML, Barrick JE, Lennon JT (2015) Rapid evolution in marine cyanobacteria: genetic and physiological responses to phage predation and resource stoichiometry. American Society for Microbiology, New Orleans, Louisiana, USA. \par

\hangindent=2em Lennon JT, Jones SE (2015) Bacterial persistence during starvation: dormancy, cannibalism, and adaptation. American Society for Microbiology, New Orleans, Louisiana, USA. \par

\hangindent=2em Cummins S, Miller KI, Lennon JT (2015) Metabolic activity of the skin microbiome: is our first line of defense sleeping on the job? American Society for Microbiology, New Orleans, Louisiana, USA. \par

\hangindent=2em Skelton J, Geyer K, Lennon JT, Brown (2015) Effects of multi-level controls and symbiont interactions on the crayfish microbiome. Society for Freshwater Science, Milwaukee, Wisconsin, USA. \par

\hangindent=2em Webster KD, Rosales-Lagarde L, Sauer PE, Schimmelmann A, Lennon JT, Boston PJ (2014) Hydrogen and carbon stable isotopic compositions and concentrations of methane in cave air of Cueva de Villa Luz, Tabasco, Mexico. American Geophysical Union, San Francisco, California, USA. \par

\hangindent=2em Elsenbroek KF, Miller KI, Lennon JT, Reynolds HL (2014) Roots of diversity: do soil microbes drive the success of prairie restoration? The Science, Practice \& Art of Restoring Native Ecosystems, East Lansing, Michigan, USA. \par

\hangindent=2em Lennon JT (2014) Dormancy, dispersal, and the assembly of microbial communities. International Symposium on Microbial Ecology, Seoul, South Korea. \par

\hangindent=2em Lennon JT, Miller KI, Locey KJ (2014) Can dormancy account for patterns of microbial biogeography? Ecological Society of America, Sacramento, California, USA. \par

\hangindent=2em Muscarella ME, Locey KJ, Nevo E, Raz S, Lennon JT (2014) Microbial community assembly at Evolution Canyon: Does dormancy dilute the effects of dispersal and filtering? Ecological Society of America, Sacramento, California, USA. \par

\hangindent=2em Locey KJ, Lennon JT (2014) A macroecological investigation of the microbial “rare biosphere”. Ecological Society of America, Sacramento, California, USA. \par

\hangindent=2em Muscarella ME, Bird KC, Larsen ML, Placella SA, Lennon JT (2014) Phosphorus resource heterogeneity affects the structure and function of microbial food webs. Joint Aquatic Sciences Meeting, Portland, Oregon, USA. \par

\hangindent=2em Lennon JT, Stuart D, Kent A, Peralta AL (2014) A social-ecological framework for micromanaging microbial services. Joint Aquatic Sciences Meeting, Portland, Oregon, USA. \par

\hangindent=2em Weitz JS, Stock CA, Wilhelm SW, Bourouiba L, Buchan A, Coleman ML, Follows MJ, Fuhrman JA, Lennon JT, Middelboe M, Sonderegger DL, Suttle CA, Thingstad TF, Wilson WH, Wommack EK (2013) A multitrophic model to quantify the effects of marine viruses on microbial food webs and ecosystem processes. Aquatic Virus Workshop 7, St. Petersburg, Florida, USA. \par

\hangindent=2em Wilhelm SW, Sonderegger DL, Stock CA, Weitz JS, Suttle CA, Bourouiba L, Buchan A, Middelboe M, Coleman ML, Follows MJ, Fuhrman JA, Lennon JT, Thingstad TF, Wilson WH, Wommack KE (2013) Mapping global distributions and activity of marine viruses. Aquatic Virus Workshop 7, St. Petersburg, Florida, USA. \par

\hangindent=2em Webster KD, Schimmelmann A, Drobniak A, Mastalerz M, Etiope G, Lennon JT (2013) Methane dynamics in limestone caves. Geological Society of America, Denver, Colorado, USA. \par

\hangindent=2em terHorst CP, Lau JA, Lennon JT (2013) The relative importance of rapid evolution in plant-soil feedbacks depends on ecological context. Ecological Society of America, Minneapolis, Minnesota, USA. \par

\hangindent=2em Peralta AL, Lennon JT (2013) Legacy effects on soil microbial communities in human-dominated ecosystems. Ecological Society of America, Minneapolis, Minnesota, USA. \par

\hangindent=2em Muscarella ME, Jones SE, Lennon JT (2013) Species sorting along a subsidy gradient affects community stability. Ecological Society of America, Minneapolis, Minnesota, USA. \par

\hangindent=2em Larsen ML, Wilhelm SW, Lennon JT (2013) Nutrient stoichiometry drives eco-evolutionary feedbacks. Midwest Ecology and Evolution Conference, South Bend, Indiana, USA. \par

\hangindent=2em Lennon JT, Muscarella ME, Jones SE (2013) Bacteria and browning: implications of terrestrial carbon subsidies for aquatic ecosystems. Association for the Sciences of Limnology and Oceanography, New Orleans, Louisiana, USA. \par

\hangindent=2em Muscarella ME, Jones SE, Lennon JT (2013) Life in brown waters: Aquatic bacterial responses to increased terrestrial carbon loading. Association for the Sciences of Limnology and Oceanography, New Orleans, Louisiana, USA. \par

\hangindent=2em Romanowicz KJ, Tringe SJ, Lennon JT, Lilleskov EA (2012) Do plant functional groups alter microbial communities and soil carbon cycling in peatlands? Argonne Soils Workshop, Argonne National Laboratory, Bloomingdale, Illinois, USA. \par

\hangindent=2em Lennon JT (2012) Can dormancy theory help us retrieve rare and uncultured microbes? International Symposium on Microbial Ecology, Copenhagen, Denmark. \par

\hangindent=2em Muscarella ME, Jones SE, Lennon JT (2012) Life in brown waters: aquatic microbial community response to increased terrestrial carbon. LTER All Scientists Meeting, Estes Park, Colorado, USA. \par

\hangindent=2em Placella SA, Brodie EL, Firestone MK, Lennon JT (2012) Soil water fluctuations: microbial community responses and CO\textsubscript{2} production. American Geophysical Union, San Francisco, California, USA. \par

\hangindent=2em Placella SA, Lennon JT (2012) Microbes, moisture, and metabolic activity: Is there a soil moisture threshold for microbial activity? Long-Term Ecological Research (LTER) All Scientists Meeting, Estes Park, Colorado, USA. \par

\hangindent=2em Hall EK, Pepe-Ranney CC, Lennon JT (2012) The effect of carbon subsidies on planktonic niche partitioning and recruitment of bacteria to marine biofilms. International Symposium on Microbial Ecology, Copenhagen, Denmark. \par

\hangindent=2em Larsen ML, Wilhelm SW, Lennon JT (2012) Nutrient stoichiometry influences rapid eco-evolutionary feedbacks in marine cyanobacteria and phage. International Symposium on Microbial Ecology, Copenhagen, Denmark. \par

\hangindent=2em Lennon JT (2012) Browning of freshwater ecosystems: culprits and consequences of global change. Ecological Society of America, Portland, Oregon, USA. \par

\hangindent=2em Peralta AL, Culman SW, Sprunger S, Lennon JT, Snapp SS (2012) Microbial contributions to carbon sequestration potential in response to perenniality. Soil Science Society of America, Cincinnati, Ohio, USA. \par

\hangindent=2em Campbell CE, Larsen ML, Lennon JT, Wilhelm SW (2011) The roles of inorganic nutrients and cyanophage in shaping heterotrophic microbial diversity. Aquatic Virus Workshop, Texel, Netherlands. \par

\hangindent=2em Larsen ML, Wilhelm SW, Lennon JT (2011) Nutrient stoichiometry generates rapid eco-evolutionary feedbacks between marine cyanobacteria and their phage. Aquatic Virus Workshop, Texel, Netherlands. \par

\hangindent=2em Lennon JT, Jones SE (2011) Metagenomics of dormancy: implications for microbial biodiversity. Ecological Society of America, Austin, Texas, USA. \par

\hangindent=2em Larsen ML, Wilhelm SW, Lennon JT (2011) Eco-evolutionary dynamics of bacteria and phage in contrasting resource environments. Ecological Society of America, Austin, Texas, USA. \par

\hangindent=2em Bird KC, Lennon JT (2011) Specialist and generalist utilization of phosphorus forms by aquatic microbes: a mechanism for maintaining microbial diversity? Association for the Sciences of Limnology and Oceanography, San Juan, Puerto Rico. \par

\hangindent=2em Lennon JT (2011) Rapid response of rare microbes linked to pulses of ecosystem activity. National Cooperative Soil Survey Conference, Asheville, North Carolina, USA. \par

\hangindent=2em Larsen ML, Wilhelm SW, Lennon JT (2011) Eco-evolutionary dynamics of bacteria and virus in nitrogen- and phosphorus-limited environments. Midwest Ecology and Evolution Conference, Southern Illinois University, Carbondale, Illinois, USA. \par

\hangindent=2em Lennon JT, Jones SE, Fierer N, Aanderud ZT (2010) Rapid response of rare microbes linked to pulses of ecosystem activity. International Symposium on Microbial Ecology, Seattle, Washington, USA. \par

\hangindent=2em Jones SE, Lennon JT (2010) Microbial dormancy: theoretical expectations and a cross-ecosystem comparison. International Symposium on Microbial Ecology, Seattle, Washington, USA. \par

\hangindent=2em Suwa T, Lennon JT, Lau JA (2011) Ecological and evolutionary effects of herbicide on plant-microbe interactions. Midwest Ecology and Evolution Conference, Carbondale, Illinois, USA. \par

\hangindent=2em Lennon JT, Jones SE (2010) Browning of the waters: Do terrestrial carbon subsidies alter aquatic ecosystem stability? Ecological Society of America, Pittsburgh, Pennsylvania, USA. \par

\hangindent=2em Lau JA, Lennon JT (2010) Belowground microbial community structure influences plant evolution. Ecological Society of America, Pittsburgh, Pennsylvania, USA. \par

\hangindent=2em Bird KC, Lennon JT (2010) Specialist and generalist utilization of phosphorus forms by aquatic microbes: a mechanism for maintaining microbial diversity? Ecological Society of America, Pittsburgh, Pennsylvania, USA. \par

\hangindent=2em Lennon JT (2010) A traits-based approach for mapping the soil microbial niche. USDA Soil Processes Meeting, Washington, District of Columbia, USA. \par

\hangindent=2em Suwa T, Lennon JT, Lau JA (2010) Mutualisms in novel environments: ecological and evolutionary implications of herbicide on plant-rhizobia interactions. Ecological Society of America, Pittsburgh, Pennsylvania, USA. \par

\hangindent=2em O’Brien JM, Hamilton SK, Kinsman LE, Ostrom N, Lennon JT (2010) Mechanisms of N retention and export in a through-flow wetland. Association for the Sciences of Limnology and Oceanography, Santa Fe, New Mexico, USA. \par

\hangindent=2em Lennon JT, Jones SE (2010) Do terrestrial carbon subsidies really stabilize aquatic ecosystem functioning? Association for the Sciences of Limnology and Oceanography, Santa Fe, New Mexico, USA. \par

\hangindent=2em Jones SE, Lennon JT (2010) Dormancy maintains diversity and structures composition of microbial communities. Association for the Sciences of Limnology and Oceanography, Santa Fe, New Mexico, USA. \par

\hangindent=2em Lennon JT, Jones SE (2009) Does presence equal activity?: Contrasting RNA- and DNA-based measures of aquatic microbial communities. Ecological Society of America, Albuquerque, New Mexico, USA. \par

\hangindent=2em Lennon JT (2009) Moisture as a “master variable” of microbial diversity and function in soils. USDA Soil Processes Meeting, East Lansing, Michigan, USA. \par

\hangindent=2em Suwa T, Lau JA, Lennon JT (2009) Ecological and evolutionary effects of herbicide on plant-rhizobia mutualisms. Ecological Society of America, Albuquerque, New Mexico, USA. \par

\hangindent=2em Aanderud ZT, Lennon JT (2009) Linking soil moisture variability, metabolically active bacteria, and CO\textsubscript{2} pulses through \(^{18}\)O DNA stable-isotope probing. Soil Science Society of America, Pittsburgh, Pennsylvania, USA. \par

\hangindent=2em Lennon JT, Schoolmaster DR, Lehmkuhl B, Aanderud ZT (2009) Mapping the niche space of diverse microbial populations along an environmental gradient. American Society for Microbiology, Philadelphia, Pennsylvania, USA. \par

\hangindent=2em Jones SE, Lennon JT (2009) Does presence equal activity?: Contrasting RNA- and DNA-based measures of aquatic microbial communities. American Society for Microbiology, Philadelphia, Pennsylvania, USA. \par

\hangindent=2em Suwa T, Lau JA, Lennon JT (2009) Effects of herbicide on rhizobia: how rapid evolutionary change may influence the outcome of plant-rhizobia mutualisms. Canadian Society of Ecology and Evolution, Halifax, Nova Scotia, Canada. \par

\hangindent=2em Burgin AJ, Hamilton SK, Lennon JT, Jones SE (2009) Nitrate use by sulfur bacteria in a stratified lake. North American Benthological Society, Grand Rapids, Michigan, USA. \par

\hangindent=2em Suwa T, Lau JA, Lennon JT (2009) Rapid evolution of rhizobia in response to glyphosate application. Midwest Ecology and Evolution Conference, Lincoln, Nebraska, USA. \par

\hangindent=2em Lennon JT (2009) The browning of freshwater ecosystems: implications for food webs and function. Association for the Sciences of Limnology and Oceanography, Nice, France. \par

\hangindent=2em Kinsman LE, O’Brien J, Lennon JT, Hamilton SK (2009) High total phosphorus concentrations in organic flocculent sediments of shallow freshwater ecosystems. Association for the Sciences of Limnology and Oceanography, Nice, France. \par

\hangindent=2em Lennon JT, Aanderud ZT, Klausmeier CA (2008) Maintenance of microbial diversity in soils: assessing the importance of habitat heterogeneity and physiological stress with theory and experiments. Ecological Society of America, Milwaukee, Wisconsin, USA. \par

\hangindent=2em Aanderud ZT, Schoolmaster DR, Lennon JT (2008) Precipitation variability decreases the responsiveness of soil CO\textsubscript{2} evolution. Ecological Society of America, Milwaukee, Wisconsin, USA. \par

\hangindent=2em Lennon JT, Schoolmaster DR, Aanderud ZT (2008) Soil moisture variability: a “master variable” of microbial activity and diversity. SoilCritZone Workshop, Chania, Crete, Greece. \par

\hangindent=2em Lennon JT, Schoolmaster DR, Aanderud ZT (2008) Plants mediate the effects of soil moisture variability on soil CO\textsubscript{2} dynamics. USDA Soil Processes Meeting, Menlo Park, California, USA. \par

\hangindent=2em Lennon JT, Cottingham KL (2007) Microbial productivity in variable resource environments. Ecological Society of America, San Jose, California, USA. \par

\hangindent=2em Lennon JT, Marston MF, Martiny JH (2006) Direct and indirect effects of viruses on the ecology and evolution of marine microbial food webs. International Symposium on Microbial Ecology, Vienna, Austria. \par

\hangindent=2em Lennon JT, Luna GM (2006) Diversity and metabolism of DNA consuming marine bacteria. International Symposium on Microbial Ecology, Vienna, Austria. \par

\hangindent=2em Lennon JT, Marston MF, Martiny JH (2006) Direct and indirect effects of viruses on the ecology and evolution of microbial food webs. Ecological Society of America, Memphis, Tennessee, USA. \par

\hangindent=2em Lennon JT, Marston MF, Hughes JB (2005) Ecological and evolutionary implications of viruses in marine microbial food. Gordon Research Conference in Applied and Environmental Microbiology, New London, Connecticut, USA. \par

\hangindent=2em Lennon JT, Marston MF, Hughes JB (2005) Marine viruses influence evolution, population dynamics, and nutrient cycling in experimental microbial food webs. Ecological Society of America, Montreal, Quebec, Canada. \par

\hangindent=2em Campbell E, Dawson A, Conner K, Lennon J, Faiia A, Feng X, Cottingham K (2005) Shifts in the relative importance of terrestrially versus aquatically produced carbon in lake ecosystems during the summer-to-fall transition. Ecological Society of America, Montreal, Quebec, Canada. \par

\hangindent=2em Thum RA, Lennon JT (2005) Ecological genetics of a milfoil invasion. Ecological Society of America, Montreal, Quebec, Canada. \par

\hangindent=2em Lennon JT (2005) Terrestrial DOM supply modifies carbon flow in lakes: evidence from stable isotopes and the composition of microbial communities. Association for the Sciences of Limnology and Oceanography, Salt Lake City, Utah, USA. \par

\hangindent=2em Thum RA, Lennon JT (2004) Does hybridization confer aggressive growth in the invasive milfoil, *Myriophyllum heterophyllum*? Evolution, Fort Collins, Colorado, USA. \par

\hangindent=2em Lennon JT (2003) Trophic state and plankton nutrition along a terrestrial DOM gradient in New England lakes. North American Lake Management Society, Mashantucket, Connecticut, USA. Recipient: Best student presentation award. \par

\hangindent=2em Thum RA, Lennon JT, Smagula A, Connor J (2003) Genetic identification of native, exotic and hybrid water milfoils in northern New England. North American Lake Management Society, Mashantucket, Connecticut, USA. \par

\hangindent=2em Lennon JT, Pfaff LE (2003) Microbial constraints on the flow of terrestrial subsidies in lake ecosystems. Ecological Society of America, Savannah, Georgia, USA. \par

\hangindent=2em Lennon JT (2003) Terrestrial subsidies in aquatic ecosystems: is carbon flow to higher trophic levels regulated by microbial metabolism? Cary Conference, Institute of Ecosystem Studies, Millbrook, New York, USA. \par

\hangindent=2em Lennon JT (2002) Experimental evidence that terrestrial organic matter modifies plankton metabolism. Association for the Sciences of Limnology and Oceanography, Victoria, British Columbia, Canada. \par

\hangindent=2em Saraidaridis J, Lennon JT (2002) Terrestrial carbon in lakes: bacterial production of phenol oxidase. Dartmouth College Women in Science Annual Meeting, Hanover, New Hampshire, USA. \par

\hangindent=2em Lennon JT, Smith VH, Dzialowski AR (2000) Community resistance to an invasion attempt by \textit{Daphnia lumholtzi}. Ecological Society of America, Snowbird, Utah, USA. \par

\hangindent=2em Lennon JT, Peterson BJ, Wollheim W (1999) Storage and transport of fine particulate organic matter in a phosphorus enriched river. Association for the Sciences of Limnology and Oceanography, Santa Fe, New Mexico, USA. \par

\hangindent=2em deNoyelles FJ, Wang SH, Meyer JO, Huggins DG, Lennon JT, Kolln WS, Randtke SJ (1999) Water quality issues in reservoirs: some considerations from a study of a large reservoir in Kansas. Proceedings of the 49th Annual Environmental Engineering Conference, University of Kansas, Lawrence, Kansas, USA. \par

\hangindent=2em Lennon JT, Dzialowski AR, O’Brien WJ, Smith VH (1998) Morphological plasticity and life history characteristics of *Daphnia lumholtzi* in the presence of invertebrate and vertebrate predators. 1998 Joint meeting between the Association for the Sciences of Limnology and Oceanography and the Ecological Society of America, St. Louis, Missouri, USA. \par

\hangindent=2em Lennon JT, Williams K (1998) Temperature and the invasion of an exotic cladoceran, *Daphnia lumholtzi*. Great Plains Limnological Society, Pittsburgh, Kansas, USA. \par

\hangindent=2em Lennon JT, Dzialowski AR (1998) The invasion of *Daphnia lumholtzi* into Kansas reservoirs. Kansas Academy of Sciences, Wichita, Kansas, USA. \par

\hangindent=2em Lennon JT, Boyer GL (1995) Toxin production by a cyanobacterium, *Aphanizomenon flos-aquae*, under different sources and supply of nitrogen. Northeastern Algal Symposium, Woods Hole, Massachusetts, USA. \par
} 

\section*{Non-degree Education}
\vspace{-1.25em} % Adjust this value as needed 
\noindent
\begin{longtable}{@{}p{3em}@{\hspace{1.5em}}p{0.87\textwidth}@{}}

2004 & Microbial Diversity, Marine Biological Laboratory, Woods Hole, Massachusetts, USA \\

2001 & Fundamentals of Ecosystem Ecology, Institute of Ecosystem Studies (IES), Millbrook, New York, USA \\

1998 & Advanced Zooplankton Ecology, Southwest Missouri State University, Springfield, Missouri, USA\\

1995 & Research Experience for Undergraduates (REU) at Toolik Lake LTER (Alaska) through the Marine Biological Laboratory, Woods Hole, Massachusetts, USA\\

1994 & Stream Ecology and Algal Ecology, University of Montana, Flathead Lake Biological Station, Yellow Bay, Montana, USA\\
\end{longtable}

\section*{Service}
\vspace{-0.5em}
\textnormal{\underline{Science Advisor:}}\\[-2.5em]
\begin{longtable}{@{}p{3em}@{\hspace{3.5em}}p{0.87\textwidth}@{}}
2024--     & A Global Partnership to Address Climate \& Biodiversity Crisis, American Society for Microbiology (ASM) and the International Union of Microbiological Societies (IUMS) \\
2019--     & Ivy Tech, Biology Advisory Board \\
2015--2019 & Shedd Aquarium, Aquarium Microbiome Project, Chicago, Illinois, USA \\
\end{longtable}

\vspace{-0.5em}
\textnormal{\underline{Editor:}}\\[-2.5em]
\begin{longtable}{@{}p{3em}@{\hspace{3.5em}}p{0.87\textwidth}@{}}
2021--2027 & Editorial Board, \textit{ISME Journal} (\textit{International Society for Microbial Ecology}) \\
2016--2022 & Editor, \textit{Environmental Microbiology} and \textit{Environmental Microbiology Reports} \\
2010--2020 & Associate Editor, \textit{Frontiers in Terrestrial Microbiology} \\
\end{longtable}

\vspace{-0.5em}
\textnormal{\underline{\textit{Ad hoc} journal reviewer}}\\[0.15em] 
\textit{Access Microbiology}, \textit{American Naturalist}, \textit{Applied and Environmental Microbiology}, \textit{Applied Soil Ecology}, \textit{Aquatic Microbial Ecology}, \textit{Aquatic Sciences}, \textit{Biogeochemistry}, \textit{Biogeosciences}, \textit{Biology Letters}, \textit{Biophysical Journal}, \textit{Canadian Journal of Fisheries and Aquatic Sciences}, \textit{Computational and Structural Biotechnology Journal}, \textit{Current Microbiology}, \textit{Ecography}, \textit{Ecology}, \textit{Ecological Applications}, \textit{Ecology Letters}, \textit{Ecoscience}, \textit{Ecosystems}, \textit{Eco-DAS Symposium Proceedings}, \textit{eLife}, \textit{Environmental Engineering Science}, \textit{Environmental Microbiology}, \textit{Evolution}, \textit{FEMS Microbiology Ecology}, \textit{Frontiers in Microbiology}, \textit{Fundamental and Applied Limnology (Archiv für Hydrobiolgie)}, \textit{Functional Ecology}, \textit{Genome Biology and Evolution}, \textit{Global Ecology and Biogeography}, \textit{Hydrobiologia}, \textit{Interface Focus}, \textit{International Journal of Environmental Health Research}, \textit{Journal of Arid Environments}, \textit{Journal of Biogeography}, \textit{Journal of Eukaryotic Microbiology}, \textit{Journal of Plankton Research}, \textit{Limnology \& Oceanography}, \textit{Journal of Virology}, \textit{Limnology \& Oceanography Methods}, \textit{mBio}, \textit{Microbial Ecology}, \textit{Microbes and Environments}, \textit{Molecular Biology and Evolution}, \textit{mSphere}, \textit{mSystems}, \textit{Nature}, \textit{Nature Communications}, \textit{Nature Geoscience}, \textit{Nature Microbiology}, \textit{Oecologia}, \textit{Oikos}, \textit{PeerJ}, \textit{PLOS \\Computational Biology}, \textit{PLOS Genetics}, \textit{PLOS ONE}, \textit{Philosophical Transactions of the Royal Society B}, \textit{Proceedings of the National Academy of Sciences}, \textit{Proceedings of the Royal Society B}, \textit{Science}, \textit{Royal Society Open Science}, \textit{Science of the Total Environment}, \textit{Soil Biology \& Biochemistry}, \textit{Soil Science Society of America Journal}, \textit{The ISME Journal}, \textit{Trends in Microbiology}, \textit{Viruses}

\vspace{1.5em}
\textnormal{\underline{Grant review panels:}}\\[-1.5em]
\vspace{-0.5em}
\begin{longtable}{@{}p{4em}@{\hspace{2em}}p{0.85\textwidth}@{}}
2021 & NSF Evolutionary Processes \\
2021 & NASA Space Biology Microbiology Panel \\
2021 & DOE Foundational Scientific Focus Area (FSFA) \\
2020 & U.S. Department of Energy's Office of Defense Nuclear Nonproliferation R\&D (DNN R\&D) \\
2020 & NASA Space Biology Microbial Communities, ROSBio Flight and Ground Review \\
2020 & Genome Alberta \\
2020 & Department of Energy, Early Career Research Program, Biological Systems Science Division (BSSD) of the Biological and Environmental Research (BER) Program Office \\
2019 & NSF Integrative and Organismal Systems (IOS), Integrative Ecological Physiology panel \\
2016 & DOE Foundational Scientific Focus Area (FSFA) \\
2015 & NSF Dimensions of Biodiversity \\
2011 & NSF Science and Technology Center (STC) \\
2011 & USDA NIFA Plant-Associated Microorganisms \\
2009 & NSF Ecosystems \\
2008 & NSF Ecosystems \\
2007 & NSF Ecosystems, Doctoral Dissertation Improvement Grants (DDIG) \\
2006 & NSF Ecology and Ecosystems, Doctoral Dissertation Improvement Grants (DDIG) \\
\end{longtable}

\vspace{-0.5em}
\textnormal{\underline{\textit{Ad hoc} grant reviewer}}\\[0.25em]
Austrian Science Fund (FWF), Czech Science Foundation (GA ČR), Chilean National Commission for Scientific and Technological Research (CONICYT), National Foundation of Science and Technology (FONDECYT), Human Frontier Science Program (HFSP), International Institute for Applied Systems Analysis (IIASA) Austria, Israel Science Foundation, Italian Antarctic Research Programme (PNRA), National Cave and Karst Research Institute (NCKRI), National Environmental Research Council (NERC) UK, Netherlands Organisation for Scientific Research (NWO), NSF Antarctic Organisms and Ecology Program, NSF Biological Oceanography, NSF Chemical Oceanography, NSF Earth Cube, NSF Population and Community Ecology, NSF Ecosystem Studies, NSF Integrative Organismal Systems, NSF Marine Geology and Geophysics, NSF Microbial Genome Sequencing Program, NSF Microbial Processes and Interactions/Microbial Observatories, NSF Office of \\ International Science and Engineering, NSF Population and Evolutionary Processes, NSF Research Coordination Networks in Biological Sciences, MIT Sea Grant Program, sDIV German Centre for Integrative Biodiversity Research (iDiv), University of Wisconsin-Milwaukee, Research Growth Initiative (RGI), USGS National Institutes of Water Resources, US Army Research Office (ARO), US Civilian Research and Development Foundation (CART), Woods Hole Sea Grant

\vspace{1.5em}
\textnormal{\underline{Promotion and tenure letter-writer}}\\[-2.5em]

\begin{longtable}{@{}p{4em}@{\hspace{2em}}p{0.85\textwidth}@{}}
2025 & Weizman Institute of Science \\
2024 & Stanford University, Department of Biology \\
2024 & University of Arizona, Department of Environmental Science \\
2024 & University of Haifa, Department of Evolutionary and Environmental Biology \\
2024 & Pennsylvania State University, Department of Plant Sciences \\
2024 & North Carolina State University, Department of Biological Sciences \\
2024 & The Ohio State University, Department of Microbiology \\
2024 & University of Wisconsin, Department of Soil Science \\
2024 & Marine Biological Laboratory, Ecosystems Center \\
2024 & Purdue University, Department of Food Science \\
2023 & Arizona State University, School of Life Sciences \\
2022 & University of Tennessee, Department of Ecology and Evolutionary Biology \\
2022 & University of Arizona, School of Natural Resources and the Environment \\
2022 & University of Arizona, Department of Environmental Science \\
2021 & Boston University, Department of Biology \\
2021 & West Virginia University, Division of Plant and Soil Sciences \\
2021 & University of Tennessee, Department of Biosystems Engineering and Soil Science \\
2021 & University of Wyoming, Ecosystem Science \& Management \\
2020 & Carnegie Institution for Science at Stanford University, Department of Plant Biology \\
2020 & University of Delaware, School of Marine Science and Policy \\
2020 & Purdue University, Department of Food Science \\
2020 & Macalester College, Department of Biology \\
2020 & University of Alaska, Department of Biology and Wildlife \\
2020 & University of Missouri – St. Louis, Department of Biology \\
2020 & Arizona State University, Biodesign Institute \\
2020 & American University in Cairo, Department of Biology \\
2020 & Marshall University, Department of Biological Sciences \\
2020 & University of Jyväskylä, Department of Biological and Environmental Sciences \\
2019 & Stanford University, Department of Biology \\
2019 & University of Minnesota, Department of Ecology, Evolution, and Behavior \\
2019 & Tufts University, Department of Biology \\
2019 & Pacific Northwest National Laboratory, Biological Science Division \\
2019 & Kansas State University, Division of Biology \\
2019 & Technion – Israel Institute of Technology, Department of Biology \\
2019 & Purdue University Northwest, Department of Biological Sciences \\
2019 & Diné College, School of Science, Technology, Engineering and Math \\
2018 & Ben-Gurion University of the Negev, Zuckerberg Institute for Water Research \\
2017 & University of Arizona, School of Natural Resources and the Environment \\
2017 & University of Colorado, Ecology and Evolutionary Biology \\
2017 & University of California Irvine, Department of Ecology and Evolutionary Biology \\
2016 & University of Maryland, Environmental Science and Technology Department \\
2016 & University of Arizona, The School of Plant Sciences \\
2016 & University of Hawai‘i at Mānoa, Department of Botany \\
2015 & University of Tennessee, Department of Biosystems Engineering and Soil Science \\
\end{longtable}

\vspace{-0.5em}
\textnormal{\underline{Professional societies}} \\[-2.5em]
\begin{longtable}{@{}p{4em}@{\hspace{2em}}p{0.85\textwidth}@{}}
2025--2026 & Chair, Academy Leadership Nomination Subcommittee, American Academy of Microbiology \\
2024--2025 & Scientific Program Leader, American Society for Microbiology \\
2024--2025 & Finance Committee, Ecological Society of America \\
2023--2024 & Strategic Planning Working Group, Ecological Society of America \\
2023--2025 & Audit Committee, Ecological Society of America \\
2022--2024 & Publications Committee, Ecological Society of America \\
2022--2023 & Track Leader, Climate Change, American Society for Microbiology \\
2022--2025 & Chair, Academy Scientific Task Force (ASAT) on Climate Change and Microbiology, American Academy of Microbiology (AAM) \\
2021--2024 & Committee Member, Ecological Society of America (ESA), Fellows and Early Career Fellows \\
2021--2022 & Ad-hoc Program Evaluation Committee (APEC), American Society for Microbiology \\
2020--2021 & Ex officio, Program Committee, Ecology, Evolution, and Biodiversity (EEB); American Society for Microbiology \\
2018--2020 & Track Leader, Ecology, Evolution, and Biodiversity (EEB); American Society for Microbiology \\
2017--2019 & Member, Council on Microbial Sciences (COMS), American Society for Microbiology \\
2017--2020 & Program Committee, American Society for Microbiology, Representative for Ecological and Evolutionary Science \\
2017--2020 & Chair, Microbial Ecology (N) Division, American Society for Microbiology \\
2016--2018 & Member, American Society for Microbiology, Committee for K--12 Outreach \\
2015--2016 & Member, American Society for Microbiology, Communication Committee’s Environmental Microbiology Taskforce \\
2016 & Abstract Reviewer, American Society for Microbiology General Meeting, Ecological and Evolutionary Science Track \\
2010--2011 & Chair, Microbial Ecology Section, Ecological Society of America \\
2009--2010 & Vice Chair, Microbial Ecology Section, Ecological Society of America \\
2008--2009 & Secretary, Microbial Ecology Section, Ecological Society of America \\
2011 & Tom Frost Award Committee, Ecological Society of America \\
\end{longtable}


\vspace{-0.5em}
\textnormal{\underline{University and College}} \\[-2.5em]

\begin{longtable}{@{}p{4em}@{\hspace{2em}}p{0.85\textwidth}@{}}
2024--2027 & College Research Faculty Promotion Subcommittee \\
2024       & Search committee, Faculty 100 Initiative, Synthetic Biology \\
2021--2023 & Member of the College of Arts and Sciences Faculty IT Advisory Council \\
2021--2022 & Search committee, Soil Microbiologist, O’Neill School \\
2024       & Section Associate Chair (interim); Evolution, Ecology, and Behavior (EEB) \\
2020--2023 & Section Associate Chair; Evolution, Ecology, and Behavior (EEB) \\
2018--2019 & Section Associate Chair (interim); Evolution, Ecology, and Behavior (EEB) \\
2018       & Chair, Biology Graduate Admissions Committee \\
2017       & EEB Graduate Program Director (GPD) \\
2014--2020 & Advisory committee, Center for Genomics and Bioinformatics (CGB) \\
2013--2019 & Executive committee member, IU Research and Training Preserve (IURTP) \\
2013--2016 & Member, Departmental Planning Committee (DPC) \\
2014--2019 & Faculty advisor, Ecolunch \\
2015, 2016 & Member, Biology Graduate Admissions Committee \\
2012--2014 & Member, Biology Graduate Recruiting Weekend \\
2011       & Site representative, LTER Science Council meeting, Jekyll Island, Georgia, USA \\
2006--2009 & Executive board member, Biogeochemistry Environmental Research Initiative (BERI), Michigan State University \\
\end{longtable}

\section*{Teaching and Mentorship}
\vspace{-0.5em}
\begin{longtable}{@{}p{4em}@{\hspace{2em}}p{0.85\textwidth}@{}}
2023--2025 & Mentor for Future Leaders Mentoring Fellowship (FLMF) program, American Society for Microbiology (ASM) \\
2012--     & Instructor: Microbial Ecology (BIO L472), Microbiomes: Host and Environmental Health (BIO L472), Microbiomes (Z620), Quantitative Biodiversity (Z620), Indiana University \\
2009--2012 & Co-Director: Summer Course in Microbial Metagenomics, Michigan State University \\
2007--2012 & Instructor: Microbial Ecology (MMG 425), Biogeochemistry (MMG 426), Michigan State University \\

2014--2024 & Junior Faculty Mentoring Team Ariane Peralta, East Carolina University \\

2006--2012 & Graduate committee member, Michigan State University: Zarraz May-Ping Lee (MMG), Molly Conlin (Plant Biology), Brian Campbell (MMG), Amy Burin (Zoology), Jason Martina (Plant Biology), Lauren Kinsman (Zoology), Micaleila Dell Desotelle (Zoology), Mridul K. Thomas (Zoology), Tomomi Suwa (Plant Biology), Stephanie Miller (Zoology), Ben Roller (MMG), Keara Towery (MMG)\\

2006-- & External graduate committee member: Karl Romanowicz (Michigan Tech), Deborah Dila (Grand Valley State University), Andreea Măgălie (Georgia Tech), Brielle Hrymoc (University of Calgary), Alex Feliciano (University of Texas El Paso)\\ 

2006-- & International dissertation opponent: Sari Peura (University of Jyväskylä, Finland), Monica Ricoa (Uppsala University, Sweden), Andrea Ramirez Corona (Université de Neuchâtel), Masumi Stadler (L'Université du Québec à Montréal) \\

2006-- & Graduate committee member, Indiana University: Freddy Lee (Microbiology), Melissa Horton (Microbiology), Elise Morton (Microbiology), Geoffrey House (EEB), Elizabeth Czerwinski (Molecular and Cellular Biochemistry), Brian Steidinger (EEB), Kimberly Elsenbroek (EEB), Kevin Webster (Geology), Maja Šljivar (EEB), Steve Kannenberg (EEB), Ali McCully (Microbiology), Ryan Fritts (Microbiology), Brianna Whittaker (EEB), Maureen Onyeziri (Microbiology), Natalie Christian (EEB), Alex Strauss (EEB), Erik Parker (EEB), Savannah Bennett (EEB), Ian Barton (Microbiology), Jeffrey Mazny (Microbiology), Michelle Benavidez (Anthropology), Katie Biedel (EEB), Mackenzie Caple (EEB), Lana Bolin (EEB), Andrea Phillips (Education), Brittany Herrin (Microbiology), Chelsea Parker (Statistics), Olivia Sheff (Microbiology), Joshua Jones (EEB), Turner DeBlieux (EEB), Young Oh (EEB), Madelynn Spencer (Microbiology), Logan Geyman (Microbiology), Andrea Shirdon (EEB), Richard Hull (EEB), Elaine Hoffman (EEB), Thomas Zambiasi (EEB) \\

2004--2005 & Teaching Certificate Program, Harriet W. Sheridan Center for Teaching and Learning, Brown University \\

2003       & Teaching Assistant, Foreign Studies Program: Ecology of Tropical Ecosystems (10-week course in Costa Rica and Jamaica), Dartmouth College \\

2002       & Women in Science Program (WISP) Mentor, Dartmouth College \\

1997--     & Trained dozens of undergraduate students in ecological, evolutionary, and microbiological research \\
\end{longtable}

\section*{Professional Society Membership}
\vspace{-0.5em}
\noindent Applied Microbiology International (AMI)\\[0.5em]
\noindent Ecological Society of America (ESA)\\[0.5em]
\noindent International Society for Microbial Ecology (ISME)\\[0.5em]
\noindent American Society for Microbiology (ASM)\\[0.5em]
\noindent International Society for the Viruses of Microorganisms (ISVM)\\[0.5em]
\noindent American Association for the Advancement of Science (AAAS)\\

\section*{Academic Advisors}
\vspace{-0.5em}
\noindent Jennifer B. Hughes Martiny, Brown University (Postdoc)\\[0.5em]
\noindent Kathryn L. Cottingham, Dartmouth College (Ph.D.)\\[0.5em]
\noindent Val H. Smith, University of Kansas (Masters)\\[0.5em]
\noindent Charles A. S. Hall, SUNY College of Environmental Science and Forestry (BS)\\

\section*{Academic Advisees}
\vspace{-0.5em}

\textnormal{\underline{Postdocs}}\\[-2.5em]
\begin{longtable}{@{}p{4em}@{\hspace{2em}}p{0.85\textwidth}@{}}
2024--     & Emma Bueren \\
2023--     & Jipeng Luo \\
2021--2024 & Canan Karakoç (Research Scholar, Georgia Institute of Technology)\\
2021--2024 & John McMullen (Data Scientist - Microbial Genomics, Bayer)\\
2017--2023 & Daniel Schwartz (Microbiologist, DSM) \\
2018--2019 & Jordan Bird (Bioinformatics Scientist at Battelle) \\
2015--2020 & Megan Behringer (Assistant Professor, Vanderbilt University) \\
2014--2018 & Ken Locey (Data Scientist, Rush University Medical Center) \\
2012--2014 & Ariane Peralta (Professor, East Carolina University) \\
2011       & Ed Hall (Associate Professor, Colorado State University) \\
2011--2012 & Sarah Placella (CEO, Root Applied Sciences) \\
2008--2010 & Stuart Jones (Executive Director, Annis Water Resources Institute) \\
2007       & Evan Kane (Professor, Michigan Technological University) \\
2007--2009 & Zachary Aanderud (Professor, Brigham Young University) \\
\end{longtable}

\vspace{1em}
\textnormal{\underline{Graduate Students}} \\[-2.5em]

\begin{longtable}{@{}p{4em}@{\hspace{2em}}p{0.85\textwidth}@{}}
2025--     & Jitul Bora (Ph.D., Biology, Indiana University) \\
2024--     & Anna Lennon (Ph.D., Biology, Indiana University) \\
2024--     & El Park (Ph.D., Biology, Indiana University) \\
2022--     & Joy O’Brien (Ph.D., Biology, Indiana University) \\
2022       & Jasmine Ahmed (M.S., Biotechnology, Indiana University) \\
2020       & Chian Jung Chen (M.S., Biotechnology, Indiana University) \\
2019--2023 & Patrick Wall (Ph.D., Complex Networks and Systems, Indiana University) \\
2019--2022 & Ford Fishman (Ph.D., Biology, Indiana University) \\
2017--     & Emmi Mueller (Ph.D., Biology, Indiana University) \\
2016--2021 & Roy Moger-Reischer (Ph.D., Biology, Indiana University) \\
2015--2018 & Venus Kuo (M.S., Biology, Indiana University) \\
2014--2020 & William Shoemaker (Ph.D., Biology, Indiana University) \\
2014--2020 & Nathan Wisnoski (Ph.D., Biology, Indiana University) \\
2013--2017 & Kevin Webster (Ph.D., Geology, Indiana University, co-advisor) \\
2010--2016 & Mario Muscarella (Ph.D., Biology, Indiana University) \\
2009--2016 & Megan Larsen (Ph.D., Biology, Indiana University) \\
2008--2012 & Kali Bird (M.S., Microbiology and Molecular Genetics, Michigan State University) \\
\end{longtable}

\end{document}
