\documentclass[11pt]{article}  % Required as first line

% Page and layout settings
\usepackage[letterpaper, top=1in, bottom=1in, left=1in, right=1in]{geometry}
\linespread{1.3}  % Slightly increase line spacing
\setlength{\parindent}{0pt}  % Removes paragraph indentation
\renewcommand{\arraystretch}{1.3}  % More space between table rows

% Font and encoding
\usepackage[utf8]{inputenc}  % For pdflatex
\usepackage[T5]{fontenc}     % For Vietnamese support

% Hyperlinks and URLs
\usepackage{hyperref}
\usepackage{url}

% Text formatting and layout
\usepackage{parskip}         % Adds spacing between paragraphs, no indent
\usepackage[none]{hyphenat}  % Disable hyphenation

% Tables
\usepackage{array}
\usepackage{tabularx}
\usepackage{longtable}
\usepackage{ltablex}
\keepXColumns

% Lists
\usepackage{enumitem}
\setlist[itemize]{noitemsep, topsep=0pt}

% Citations (optional — used only if you call \cite or \bibliography)
\usepackage{natbib}
% \usepackage{etaremune}  % Removed for non-numbered references

% Rename bibliography section if needed
\renewcommand{\refname}{Publications}


\begin{document}

% Ce
% Centered header
\begin{center}
  {\LARGE \textbf{Jay-Terrence Lennon}}\\[0.5em]
  Department of Biology, Indiana University, Bloomington, Indiana 47405, USA \\
  %Phone: (812) 856-0962 \\
  Email: \href{mailto:lennonj@iu.edu}{lennonj@iu.edu} \\
  Lab website: \url{https://lennonlab.github.io} \\
  Lab wiki: \url{https://lennon.bio.indiana.edu} \\
  Google Scholar: \url{https://goo.gl/qx4hHR}
\end{center}

%\vspace{1em}

\section*{Education}
\noindent
\begin{tabular}{@{}l@{\hspace{3em}}l@{\hspace{3em}}l@{\hspace{3em}}l@{}}
1995 & B.S.    & Environmental Forest Biology     & SUNY-ESF at Syracuse \\
1999 & M.A.    & Ecology and Evolutionary Biology  & University of Kansas \\
2004 & Ph.D. & Ecology and Evolutionary Biology  & Dartmouth College \\
\end{tabular}

\section*{Professional experience}
\vspace{-1.25em} % Adjust this value as needed 
\noindent

\begin{tabularx}{\textwidth}{@{}l@{\hspace{2em}}X@{}}
2023         & Visiting Professor, Goethe University, Frankfurt, Germany \\
%2023         & Short-term Visiting Professor, ETH Zürich, Centre for Origin and Prevalence of Life \\
%2020--2024   & Faculty, Complex Networks and Systems, Indiana University \\
2018--2022   & Faculty, Microbial Diversity Course, Marine Biological Laboratory, Woods Hole \\
2016--       & Professor, Indiana University, Department of Biology \\
%2016--2017   & Whitman Center Associate, Marine Biological Laboratory, Woods Hole \\
%2016--2017   & Visiting Professor, Montana State University, Department of Microbiology and Immunology \\
2012--2016   & Associate Professor, Indiana University, Department of Biology\\
2012         & Associate Professor, W.K. Kellogg Biological Station, Department of Microbiology and Molecular Genetics, Michigan State University \\
%2011--2016   & Ad hoc Graduate Faculty, Michigan Technological University \\
%2008--2012   & Adjunct Professor, Plant Biology Department, Michigan State University \\
2006--2012   & Assistant Professor, W.K. Kellogg Biological Station, Department of Microbiology and Molecular Genetics, Michigan State University \\
2004--2006   & Postdoctoral Research Associate, Brown University, Department of Ecology and Evolutionary Biology \\
\end{tabularx}
\vspace{-1.9em} % Adjust this value as needed
\section*{Honors}
\vspace{-1.25em} % Adjust this value as needed 
\noindent

\begin{tabularx}{\textwidth}{@{}l@{\hspace{2em}}X@{}}
2025--2028  & Chair, Scientific Unit on Applied and Environmental Microbiology, American Society for Microbiology (ASM) \\
%2025 --     & Steering committee, International Union for Conservation (IUCN), Microbial Conservation Specialist Group (MCSG)\\
%2024        & Highly Cited Author, American Society for Microbiology (ASM) \\
%2024        & Soil Stars, Applied Microbiology International (AMI) \\ 
2023        & Humboldt Prize, Alexander von Humboldt Foundation \\
2022--2025  & Governing Board, Ecological Society of America (ESA) \\
2022--2027  & Chair, Climate Change Task Force, American Academy of Microbiology (AAM) \\
2021        & Fellow, Ecological Society of America (ESA) \\
2020--2022  & Distinguished Lecturer, American Society for Microbiology (ASMDL) \\
2020--2026  & Governor, American Academy of Microbiology (AAM) \\
2019        & Fellow, American Academy of Microbiology (AAM) \\
2019--2024  & Highly Cited Researcher, Clarivate, Cross-Field \\
2018        & Fellow, American Association for the Advancement of Science (AAAS) \\
2012        & Kavli Fellow, National Academy of Sciences \\
%2004        & USDA National Research Initiative (NRI) Postdoctoral Fellowship Award \\
%2004        & Hannah T. Croasdale Graduate Scholar Award, Dartmouth College \\
%2004        & Milton L. Shifman Endowed Scholarship, Marine Biological Laboratory \\
%2004        & Albert Cass Fellowship, The Rockefeller University \\
%2004        & Nathan Jenks Biology Award, Dartmouth College \\
%2003        & Best student presentation, North American Lake Management Society National Meeting, Mashantucket, CT \\
%2002        & NSF Doctoral Dissertation Improvement Grant (DDIG) \\
%1999--2004  & Dartmouth Fellowship, Dartmouth College \\
%1995        & Undergraduate honors: \textit{Magna Cum Laude}; President’s List; Alpha Sigma Xi, SUNY-ESF \\
%1992        & Outstanding history student, SUNY Oswego \\
\end{tabularx}
\vspace{-1.9em} % Adjust this value as needed
%\newpage
\section*{Select Publications}
%(10 most relevant of >150. For full list, see: \href{https://scholar.google.com/citations?hl=en&user=d-hWatsAAAAJ}{Google Scholar}, \href{https://www.ncbi.nlm.nih.gov/myncbi/jay.lennon.1/bibliography/public/}{PubMed}, or \href{https://www.ncbi.nlm.nih.gov/myncbi/jay.lennon.1/bibliography/public/}{Lab Website})
\vspace{-0.05em}
\begin{itemize}[leftmargin=*, label={}, itemsep=1em]

\item Lennon JT, Lehmkuhl BK, Chen L, Illingworth M, Kuo V, Muscarella ME (2025) \\Resuscitation-promoting factor (Rpf) terminates dormancy among diverse soil bacteria.\\
\textit{mSystems} 10: 01517-24. \href{https://lennonlab.github.io/assets/publications/Lennon_etal_2025b.pdf}{(pdf)}

\item Moger-Reischer RZ, Glass JI, Wise KS, Sun L, Bittencourt DMC, Lehmkuhl BK, Schoolmaster DR Jr, Lynch M, Lennon JT (2023) Evolution of a minimal cell. \textit{Nature} 620: 122--127. \href{https://lennonlab.github.io/assets/publications/Moger-Reischer_etal_2023.pdf}{(pdf)}

\item Schwartz DA, Shoemaker WR, Măgăliee A, Weitz JS, Lennon JT (2023) Bacteria-phage coevolution with a seed bank. \textit{ISMEJ} 17: 1315–1325. \href{https://lennonlab.github.io/assets/publications/Schwartz_etal_2023b.pdf}{(pdf)}

\item Schwartz DA, Rodriguez-Ramos J, Shaffer M, Flynn F, Daly R, Wrighton KC, Lennon JT (2023) Human-gut phages harbor sporulation genes. \textit{mBio} e0018223. \href{https://lennonlab.github.io/assets/publications/Schwartz_etal_2023a.pdf}{(pdf)}

%\item Schwartz DA, Lekmkuhl BK, Lennon JT (2022) Phage-encoded sigma factors alter bacterial dormancy. \textit{mSphere} e00927-22. \href{https://lennonlab.github.io/assets/publications/Schwartz_etal_2022.pdf}{(pdf)}

\item Shoemaker WR Jones SE Muscarella ME Behringer MG Lehmkuhl BK Lennon JT (2021) Microbial population dynamics and evolutionary outcomes under extreme energy-limitation. \textit{PNAS} 118: e2101691118. \href{https://lennonlab.github.io/assets/publications/Shoemaker_etal_2021b.pdf}{(pdf)}

\item Lennon JT, Muscarella ME, Placella SA, Lehmkuhl BK (2018) How, when, and where relic DNA biases estimates of microbial diversity. \textit{mBio} 9: e00637-18. \href{https://lennonlab.github.io/assets/publications/Lennon_etal_2018.pdf}{(pdf)}

\item Locey KJ, Lennon JT (2016) Scaling laws predict global microbial diversity. \textit{PNAS} 113: 5970–5975. \href{https://lennonlab.github.io/assets/publications/Locey_Lennon_2016.pdf}{(pdf)}

\item Lau JA, Lennon JT (2012) Rapid responses of soil microorganisms improve plant fitness in novel environments. \textit{PNAS} 109: 14058–14062. \href{https://lennonlab.github.io/assets/publications/Lau_Lennon_2012.pdf}{(pdf)}

\item Jones SE, Lennon JT (2010) Dormancy contributes to the maintenance of microbial diversity. \textit{PNAS} 107: 5881-5886. \href{https://lennonlab.github.io/assets/publications/Jones_Lennon_2010.pdf}{(pdf)}

\end{itemize}

%\vspace{0.5 em} % Reduce space
%\section*{D. Grants and funding}

%(Active over last 10 years. For complete list of awards totaling $>$~\$31.6 M, see: \href{https://lennonlab.github.io/assets/docs/Lennon_CV.pdf}{full-length CV})

% \vspace{-1.25em}
% \noindent
% \begin{tabularx}{\textwidth}{@{}l@{\hspace{2em}}X@{}}
% 2025--2026	& \textit{Pending}: National Science Foundation (NSF) “Conference: A unifying framework for dormancy across scales in natural, managed, and engineered ecosystems” Co-PI, \$99,000\\
% 2025--2030	& \textit{Pending}: National Institutes of Health (NIH) “Cellular dormancy and virus entrapment” PI, \$2,161,566\\
% 2025--2028 & Department of Natural Resources (DNR) "Development and testing of microbial mitigation of coalbed methane emissions in Indiana: A geo-microbial-engineering approach” Co-PI, \$287,295 \\
% 2023--2024 & Department of Defense (DoD) “Molecular-based methods for the mark-recapture of microorganisms” PI, \$225,000 \\
% 2023 & Humboldt Research Fellowship, Alexander von Humboldt Foundation, Germany, €60,000 (\$68,780) \\
% 2022--2025 & Army Research Office (ARO) “Complexity of the gut microbiome: a quantitative and experimental approach” PI, \$449,862 \\
% 2020--2025 & National Science Foundation (NSF) “BII-Implementation: Multiscale interactions of nested genomes in symbiosis: From genes to global change” Co-investigator, \$12.5M \\
% 2020--2023 & National Science Foundation (NSF) “Collaborative Research: BEE: A dormancy refuge in host-parasite eco-evolutionary dynamics” PI, \$976,617 \\
% 2020--2025 & National Science Foundation (NSF) “CNH2-L: Resilience to drought or a drought of resilience? The potential for interactions and feedbacks between human adaptation and ecological adaptation” Co-PI, \$1,599,684 \\
% 2020--2023 & National Aeronautics and Space Administration (NASA) “Energy limitation and the evolution of microbial dormancy” PI, \$733,792 \\
% 2020 & Army Research Office (ARO) "Mechanistic insight into bacterial metabolism from long-term evolution experiments" PI, \$550,000 \\
% 2018--2020 & Army Research Office (ARO) “Microbial evolution: linking genes, phenotype, and fitness in bacterial populations” PI, \$499,330 \\
% 2018--2019 & Indiana University Collaborative Research Grant (IUCRG). “Complexity of the gut microbiome: an experimental approach” PI, \$74,990 \\
% 2017--2018 & Army Research Office (ARO) "Connecting phenotype to genotype in evolved prokaryotic populations.” Co-PI, \$197,390 \\
% 2015--2022 & Department of Defense, Multidisciplinary University Research Initiatives (MURI) Program, “Mechanisms of prokaryotic evolution” PI, \$6,248,455 \\
% 2015--2020 & National Science Foundation (NSF) “Dimensions: Collaborative Research: Microbial seed banks: processes and patterns of dormancy-driven biodiversity” PI, \$1,997,144 \
% \end{tabularx}



\end{document}