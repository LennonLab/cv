\documentclass[11pt]{article}

% Page layout
\usepackage[margin=1in]{geometry}
\usepackage{setspace}
\setstretch{1.15}

% Fonts and language
\usepackage[T1]{fontenc}
\usepackage[utf8]{inputenc}
\usepackage{lmodern}
\usepackage[english]{babel}

% Paragraph spacing
\usepackage{parskip}

% Hyperlinks
\usepackage[hidelinks]{hyperref}

% Section styling (if needed)
\usepackage{titlesec}
\titleformat{\section}{\large\bfseries}{}{0em}{}[\titlerule]

% No page numbers
\pagestyle{empty}

\begin{document}

\section*{Biography: Jay T. Lennon}

Lennon is a Professor in the Department of Biology at Indiana University and former Chair of the Evolution, Ecology, and Behavior (EEB) Section. He teaches a computationally focused graduate course in \textit{Quantitative Biodiversity} and an undergraduate course on \textit{Microbiomes: Host and Environmental Health}, and was an instructor for the Microbial Diversity course at the Marine Biological Laboratory (MBL) in Woods Hole. Previously, Lennon was an Assistant and Associate Professor in the Department of Microbiology \& Molecular Genetics and the W.K. Kellogg Biological Station (KBS) at Michigan State University. He completed postdoctoral training at Brown University and earned a Ph.D. at Dartmouth College, a Master’s degree at the University of Kansas, and a Bachelor’s degree at the SUNY College of Environmental Science and Forestry (ESF) at Syracuse University.

Lennon’s research team is motivated by the ecological and evolutionary processes that generate and maintain microbial biodiversity. In turn, they investigate the implications of diversity for the stability and functioning of ecosystems using molecular biology, mathematical modeling, data synthesis, laboratory experiments, and field work in a wide range of habitats. His group has shed light on the role of functional traits for predicting community dynamics. They have documented the importance of dormancy, whereby individuals can enter a reversible state of reduced metabolic activity. Recent work has shown how the resulting seed banks lead to the emergence of complex phenomena. His group also integrates microbial life forms with other taxa across the tree of life to test macroecological theory. Such efforts have unveiled scaling laws which predict Earth is home to one trillion ($10^{12}$) species.

Lennon has been engaged in society leadership....

Lennon has published more than 150 papers with competitive awards from the National Science Foundation (NSF), the National Aeronautics and Space Administration (NASA), the Department of Defense, (DoD), and the United States Department of Agriculture (USDA). Among other recognitions, Lennon has been elected to the American Association for the Advancement of Science (AAAS), the American Academy of Microbiology (AAM), and the Ecological Society of America (ESA). He is a recipient of the Humboldt Prize from the Alexander von Humboldt Foundation and was selected as a Kavli Fellow by the National Academy of Sciences. 


\end{document}
